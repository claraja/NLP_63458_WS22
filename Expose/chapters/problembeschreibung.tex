% !TEX root =expose.tex
\chapter{Problembeschreibung}

Es soll eine Konsolenapplikationen entwickelt werden, die medizinische Texte analysiert und das darin enthalten Wissen strukturiert in einer XML-Datei dokumentiert oder mittels eines RDF-Modells visualisiert. Die Konsolenapplikation soll in Python programmiert werden, wobei für die Textanalyse Klassen und Methoden der Open-Source-Bibliothek spaCy o.ä. genutzt werden sollen.

Die medizinischen Texte enthalten z.B. Aussagen zu Krankheiten (z.B. Depression) und zählen deren Symptome (z.B. Motivationsverlust) auf. Aufgabe der Konsolenapplikation ist es dann, Schlüsselbegriffe im Text zu finden und miteinander in Bezug zu setzen, indem z.B. Symptome den ihnen zugrunde liegenden Krankheiten zugeordnet werden. Die von spaCy oder anderen NLP-Bibliotheken zur Verfügung gestellte Funktionalität besteht aus einer Pipeline von Analyse-Komponenten, die nacheinander auf den zu analysierenden Text angewendet werden. Diese Komponenten dienen dazu, Texte in einzelne Sätze zu zerlegen und die Sätze grammatikalisch zu analysieren.

Neben der grammatikalischen Analyse besteht eine wesentliche Aufgabe von spaCy oder anderen NLP-Bibliotheken darin, Schlüsselbegriffe zu finden und nach Möglichkeit zu kategorisieren. Diese Art der Analyse wird allgemein als Named Entity Recognition (NER) bezeichnet. Bei spaCy übernimmt diese Aufgabe der regelbasierte EntityRuler oder der auf statistischen Modellen basierende EntityRecognizer. Eine weitere Komponente ist der EntityLinker, mit dem Begriffe eindeutig den in einer Wissensbasis gespeicherten Entitäten zugeordnet werden können. Auch der EntityLinker basiert auf statistischen Modellen und wird mit Beispielsätzen trainiert.

Die Standard-NER Funktionalität von spaCy erkennt medizinische Begriffe nur unzureichend. Es ist daher notwendig, eine Datenbank mit medizinischen Begriffen und Kategorien zusammenzustellen, die für das Training der NER-Komponente verwendet werden kann.

Die auf dem Campus des National Institute of Health (NIH) im US-Bundestaat Maryland angesiedelte National Library of Medicine (NLM) stellt mit dem Unified Medical Language System (UMLS) ein mächtiges Werkzeug für die Textanalyse zur Verfügung. Teil von UMLS ist der sogenannte Metathesaurus, der aus einer Vielzahl von Thesauri unterschiedlicher Organisationen zusammengestellt wird. Zu diesen Thesauri gehört u.a. der vom National Library of Medicine entwickelte Medical Subject Heading (MeSH)-Thesaurus. MetaMap und das weniger umfangreiche MetaMapLite sind eigene Entity-Recognition-Werzeuge der National Library of Medicine. Die zugrundeliegende Datenbasis dieser Werkzeuge sollen im Rahmen dieses Praktikums dafür verwendet werden, die Named Entity Recognition-Komponenten von spaCy zu trainieren. Aus den vom NLM zur Verfügung gestellten Begriffslisten soll eine geeignete Auswahl erfolgen, die für das Training der NER-Komponente verwendet werden kann.

Basierend auf den gefundenen Entitäten soll das Python-Programm in der Lage sein, Begriffe wie Krankheiten und Symptome richtig zuzuordnen. Hierzu muss die Struktur von Aussagesätze, Fragen, Aufzählungen und Überschriften analysiert werden und so aufbereitet werden, dass eine automatische Analyse durch NER und EntityLinker möglichst erfolgreich ist. Die Ausgabe der Konsolenapplikation soll eine XML-Datei o.ä. sein, die auch als Grundlage einer graphischen Visualisierung im RDF Format dienen kann.

