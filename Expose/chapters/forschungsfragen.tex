% !TEX root =expose.tex
\chapter{Forschungsfragen und Ziele}

Dieser Abschnitt beschreibt die Forschungsfragen, die sich aus dem in Abschnitt \ref{ch:problembeschreibung} angeführten Problemen ableiten.

%\setlist[ff]{label=FF \arabic*}
%\setlist[fz]{label=FZ \arabic*}

\begin{enumerate}[label=FF\arabic*]
\item Wie kann ein medizinischer Text vorverarbeitet werden, so dass relevante Aussagen (Überschriften, Aufzählungen, Leerzeilen und Dialoge) automatisch erkannt werden?
\item Wie kann ein medizinisches Fachvokabular über depressive Erkrankungen automatisiert in eine maschinenlesbare Form überführt werden?
\item Wie kann eine Wissensrepräsentation über einen gegebenen Text maschinenlesbar erstellt werden?
\end{enumerate}

\begin{enumerate}[label=FZ \arabic*]
\item Wie kann ein medizinischer Text vorverarbeitet werden, so dass relevante Aussagen (Überschriften, Aufzählungen, Leerzeilen und Dialoge) automatisch erkannt werden?
\item Wie kann ein medizinisches Fachvokabular über depressive Erkrankungen automatisiert in eine maschinenlesbare Form überführt werden?
\item Wie kann eine Wissensrepräsentation über einen gegebenen Text maschinenlesbar erstellt werden?
\end{enumerate}

1.	Wie kann ein medizinischer Text vorverarbeitet werden, so dass relevante Aussagen (Überschriften, Aufzählungen, Leerzeilen und Dialoge) automatisch erkannt werden?


2.	Wie kann ein medizinisches Fachvokabular über depressive Erkrankungen automatisiert in eine maschinenlesbare Form überführt werden?


3.	Wie kann eine Wissensrepräsentation über einen gegebenen Text maschinenlesbar erstellt werden?


