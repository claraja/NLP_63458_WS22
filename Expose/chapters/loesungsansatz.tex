% !TEX root =expose.tex
\section{Lösungsansatz}

Die Wissensrepräsentation medizinischer Texte soll Entitäten unterschiedlicher Kategorie miteinander in Beziehung setzen. Naheliegend ist es z.B., automatisch in einem Text Symptome und Krankheiten zu identifizieren, diese miteinander in Beziehung zu setzen und eine Wissensrepräsentation der Art

\begin{center}
loss of appetite - is a symptom of - depression
\end{center}

zu erzeugen. Dazu muss spaCy befähigt werden, nicht nur Fachbegriffe zu erkennen, sondern diese auch richtig zu kategorisieren, als z.B. Krankheit oder Symptom. Folgender Lösungsansatz ist angedacht:

MetaMapLite verwendet eine auf UMLS basierende Datenbasis mit medizinischen Fachbegriffen. Der Umfang der Datenbasis von MetaMapLite ist etwas reduziert und umfasst z.B. nur englische Fachbegriffe. Auf der MetaMapLite-Website kann das zip-Archiv

\begin{center}
\emph{public\_mm\_data\_lite\_usabase\_2022aa.zip}
\end{center}

heruntergeladen werden. In diesem Archiv befindet sich die Datei \emph{postings} in dem Unterordner

\begin{center}
\emph{\textbackslash public\_mm\_lite\textbackslash data\textbackslash ivf \textbackslash 2022AA\textbackslash USAbase\textbackslash indices\textbackslash cuisourceinfo}.
\end{center}

Die Datei \emph{postings} enthält ca. 11 Millionen englischsprachige Einträge und ist mit einer Größe von etwa 739 MB etwas handlicher als die Datenbestände von UMLS. Um zu prüfen, ob diese Datenbasis geeignet ist, ist ein Auszug von Einträgen, die für den zu analysierenden Beispieltext über Depressionen relevant sind, in der folgenden Tabelle aufgelistet.

CUI bezeichnet den Concept Unique Identifier, mit dem Synonyme gefunden werden können. Der CUI besteht nur aus dem mit dem Buchstaben `C' beginnenden Teil und ist hier ergänzt durch einen sogenannten Term-Type. Der Term-Type `PT' kennzeichnet z.B. bevorzugte Einträge. SUI bezeichnet den String Unique Identifier, mit dem gleichlautende (und damit redundante) Bezeichnungen gefunden werden können.

Die Tabelle listet nur Einträge auf, in denen der Begriff, z.B. `depression', einer in Klammern hinter dem Begriff stehenden Kategorie zugeordnet ist, hier z.B. `disease'. Über den CUI können in der Datei \emph{postings} Synonyme gefunden werden, wobei nur der mit `C' beginnende Teil relevant ist. Diese Synonyme enthalten häufig keine Kategorien in Klammern. Die Kategorien können aber über den CUI den Synonymen zugeordnet werden. Die sehr zahlreichen Synonyme bzw. alternativen Ausdrücke sind in der Tabelle nicht aufgeführt.

\begin{center}
\begin{tabular}{llll}
\hline
\textbf{CUI}	& \textbf{SUI}	& \textbf{Item} & \textbf{Source} \\
\hline
FNC0232933 &	S3225525 & \parbox[t]{5cm}{Abnormal menstrual cycle (finding)} &	SNOMEDCT\_US \\
PTC0424569 &	S3221759 & \parbox[t]{5cm}{Circumstances interfere with sleep (disorder)} &	SNOMEDCT\_US \\
YC0009806 & S3235964 & Constipation (finding) &	SNOMEDCT\_US \\
LAC3845528 &	S14560529 &\parbox[t]{5cm}{Depressed mood (e.g., feeling sad, tearful)} & LNC \\
SYC0344315 &	S3252744 &	Depressed mood (finding) &	SNOMEDCT\_US \\
GTC0011570 &	S1431189 &	depression (disease) & AOD \\
FNC2939186 &	S3264511 &	Disturbance in mood (finding) & SNOMEDCT\_US \\
FNC1288289 &	S3312713 &	Fearful mood (finding) & SNOMEDCT\_US \\
FNC0150041 &	S3313279 &	Feeling hopeless (finding) &	SNOMEDCT\_US \\
FNC0022107 &	S3313282 &	Feeling irritable (finding)	 & SNOMEDCT\_US \\
FNC0424000 &	S3313310 &	Feeling suicidal (finding) &	SNOMEDCT\_US \\
ETC0917801 &	S3373158 &	Insomnia (disorder) &	SNOMEDCT\_US \\
PTC0015672 &	S3386372 &	Lack of energy (finding) &	SNOMEDCT\_US \\
FNC2981158 &	S3386389 &	Lack of libido (finding) & SNOMEDCT\_US \\
FNC1971624 &	S3397609 &	Loss of appetite (finding) & SNOMEDCT\_US \\
FNC0178417 &	S3397620 &	\parbox[t]{5cm}{Loss of capacity for enjoyment (finding)} & SNOMEDCT\_US \\
FNC0456814 &	S3397668 &	Loss of motivation (finding)  & SNOMEDCT\_US \\
FNC0679136 &	S3398077 &	Low self-esteem (finding) & SNOMEDCT\_US \\
PTC5444612 &	S20749480 & mood (physical finding) & MTH \\
FNC2945580 &	S3485453 &	Poor self-esteem (finding) &	SNOMEDCT\_US \\
FNC0235160 &	S3513783 &	Restless sleep (finding) & SNOMEDCT\_US \\
PTC0233481 &	S3620195 &	Worried (finding) & SNOMEDCT\_US \\
\hline
\end{tabular}
\end{center}

In der Tabelle finden sich drei Kategorien, `disease', `disorder' und `finding' (Die \emph{postings}-Datei enthält noch weitere wie z.B. `situation' oder `procedure'). Hier bietet sich an, die Begriffe und Kategorien dieser Datenbasis für das Trainieren des Entity-Recognizers oder Entity-Rulers von spaCy zu verwenden. Die Zuordnung `finding' bedeutet eigentlich Befund, kann hier aber auch als Symptom verstanden werden. `disorder', also Störung, kennzeichnet in den meisten Fällen Krankheiten. Der Begriff `insomnia' findet sich nicht im zu analysierenden Text, dort ist stattdessen von `disturbed sleep' die Rede ist. Hier muss versucht werden, über Synonyme eine richtige Kategorisierung zu erreichen. Die meisten Einträge stammen von der Datenbank SNOMED CT (Systematized Nomenclature of Medicine Clinical Terms), die seit 2003 Teil des UMLS Metathesaurus ist. 

Der Inhalt der Datei \emph{postings} muss aufbereitet (und ggf. auch gekürzt) werden, so dass die Daten für das Training des Entity-Recognizers verwendet werden können. Es muss untersucht werden, wie basierend auf der Wörterliste geeignete Trainingsdaten erzeugt werden können, etwa durch automatisch erzeugte Beispielsätze. Alternativ kann der Entity-Ruler zum Einsatz kommen, dem die Wörterliste einfach übergeben werden kann und dessen Funktion vorhersagbarer ist als die des Entity-Recognizers.

Der zu analysierende Text muss durch ein Python-Programm aufbereitet werden, mit dem Ziel, dass die NLP-Komponenten der Prozesspipeline des NLP-Frameworks, möglichst erfolgreich arbeiten. Ein Vorgehen könnte beispielsweise sein, Aufzählungen in mehrere vollständige Sätze zu zerlegen, so dass möglichst sinnvolle vollständige Sätze entstehen, die eine Krankheit und ein Symptom enthalten, so dass die automatische Textanalyse nicht überfordert wird, Krankheit und Symptome zusammenzubringen. Hierbei werden die von \emph{spaCy} zur Verfügung gestellten Pipeline-Komponenten wie der Lemmatizer und Parser genutzt.

Es ist zu prüfen, inwieweit der \emph{Dependency Parser} von \emph{spaCy} genutzt werden kann, um Abhängigkeiten und Subjekt-Prädikat-Objekt-Beziehungen in Sätzen zu erkennen, die für die Erstellung des RDF-Graphen genutzt werden können.

Idealerweise entsteht jedoch bereits durch die NER ein Text mit Aussagesätzen, die jeweils eine Krankheit und ein oder mehrere Symptome enthalten. Für die Wissensrepräsentation soll der Entity-Linker eingesetzt werden. Normalerweise wird der Entity-Linker verwendet, um (mehrdeutige) Entitäten, die vom Entity-Recognizer oder Entity-Ruler gefunden werden, einer eindeutigen Entität zuzuordnen (z.B. einem Wikipedia-Eintrag). Der Entity-Linker wird trainiert, so dass die Zuordnung Kontext-basiert erfolgt.

Für die Erstellung der Wissensrepräsentation kann der Entity-Linker auf unorthodoxe Weise verwendet und mit dieser Komponente die Wissensrepräsentation aufgebaut werden. Wesentlich ist hierbei die vorhandene Datenstruktur der Wissensbasis des Entity-Linkers. In dieser werden durch das Python-Programm als zusammengehörend erkannte Krankheiten und Symptome gespeichert. Dabei können Symptome mehreren Krankheiten zugeordnet werden. Die nach Analyse eines Textes in der Wissensbasis gespeicherten Daten, werden dann als XML-Datei oder RDF-Modell ausgegeben. Alternativ zum Entity-Linker kann eine eigene Datenstruktur zum Zwischenspeichern der Wissensrepräsentation genutzt werden.

Ein Vorteil des Der Entity-Linkers ist, dass er mit Sätzen aus den zu analysierenden Texten trainiert werden kann. Eine kontinuierlich wachsende Wissensbasis (durch zahlreiche automatische Textanalysen) vorausgesetzt, erlaubt dies dem Entity-Linker, im Laufe der Zeit in beliebigen Texten Kontext-basiert einem Symptom die richtige Krankheit zuzuordnen. Analysiert man einen Text dann mit dem Entity-Recognizer und danach mit dem Entity-Linker, dann entsteht automatisch die Wissensrepräsentation, wobei etwa die Entität `lack of energy' vom Entity-Recognizer gefunden wird und als `Symptom' kategorisiert wird und anschließend vom Entity-Linker der Krankheit `depression' zugeordnet wird, sofern sich dies aus dem Kontext ergibt.

\vspace{1cm}

Arbeitspakete:
\begin{itemize}
    \item Manuelle Erstellung einer Wissensrepräsentation basierend auf dem zu analysierenden Text.
    \item Identifizierung eines medizinischen Vokabulars, das zum Training der NER-Komponente verwendet werden kann.
    \item Untersuchung, auf welche Weise die Trainingsdaten für möglichst gute Ergebnisse der NER-Komponente aufbereitet werden müssen (z.B. durch automatisch erzeugte Beispielsätze). Alternativ Verwendung des Entity-Rulers.
    \item Entwicklung einer Programm-Komponente, die die zu analysierenden Texte vorverarbeitet, so dass die NLP-Pipeline möglichst effizient arbeitet.
    \item Entwicklung einer Programm-Komponente, die basierend auf den vom NER gefundenen Entitäten zusammengehörende Entitäten (etwa Symptom und Krankheit) identifiziert und eine Wissensbasis aufbaut, ggf. unter Ausnutzung der Wissensbasis des Entity-Linkers.
    \item Entwicklung einer Programm-Komponente, die die Wissensbasis als RDF/XML-Datei ausgibt.
\end{itemize}



