% !TEX root =expose.tex
\section{Motivation}

Die spezielle Herausforderung der Praktikumsaufgabe liegt darin, automatisiert Zusammenhänge zwischen Entitäten medizinischer Texte zu finden und diese Zusammenhänge zu repräsentieren. Dem zugrunde liegt das Ziel des Dissertationsprojektes von Stephanie Heidepriem, zu erforschen, wie ein textbasierter Chatbot zur Unterstützung 
psychisch kranker Menschen entwickelt werden kann. 

Das Dissertationsprojekt  ist unter anderem an das Projekt \emph{MENHIR} angelehnt \cite{menhir}. Dieses Projekt dient der Erforschung von Konversationstechnologien, die psychisch kranke Menschen unterstützen sollen. In diesem Zusammenhang wird auch 
ein MENHIR-Chatbot entwickelt, der personalisierte Unterstützung und hilfreiche Bewältigungsstrategien bieten soll. 

Das Forschungsprojekt \emph{STop Obesity Platform} beschäftigt sich mit der Extraktion von Wissen unter anderem aus Chatbots, das anschließend aufbereitet und mit weiteren Informationen kombiniert werden soll. Dieses Wissen soll medizinischem Fachpersonal zur Verfügung gestellt werden und außerdem Menschen mit Adipositas dabei helfen, eine gesündere Ernährung einzuhalten \cite{stopobesity}.


Es ist aber auch denkbar, dass die Problemstellung der Praktikumsaufgabe auch darüber hinaus für weitere Anwendungen interessant ist und mögliche Ergebnisse in verschiedenen Gebieten eingesetzt werden können, in denen Informationen in (medizinischen) Texten zueinander in Zusammenhang gebracht werden sollen.

Es existieren bereits Methoden, die in der Lage sind Entitäten, also wichtige Objekte wie Personen, Organisationen oder Orte, automatisiert aus Texten zu extrahieren. Diese als \emph{Named Entity Recognition} bezeichneten Verfahren gehören zum Bereich des \emph{Natural Language Processing} und werden seit circa 30 Jahren entwickelt \cite{trends_in_ner}. 

Mit den \emph{Medical Subject Headings} \cite{mesh} steht ein auf den medizinischen Bereich spezialisiertes Vokabular zur Verfügung, das regelmäßig aktualisiert wird. Dieses Vokabular wurde von der \emph{United States National Library of Medicine} erstellt. Diese Bibliothek stellt außerdem den \emph{Named-Entity Recognizer}
\glqq MetaMapLite\grqq{} zur Verfügung, der auch an weitere Nutzungszwecke angepasst und erweitert werden kann \cite{metamaplite}. 

