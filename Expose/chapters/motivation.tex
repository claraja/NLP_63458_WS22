% !TEX root =expose.tex
\chapter{Motivation}

Die spezielle Herausforderung unserer Praktikumsaufgabe liegt darin, automatisiert Zusammenhänge zwischen Entitäten 
medizinischer Texte zu finden und diese Zusammenhänge zu repräsentieren.
Dem zugrunde liegt das Ziel des Dissertationsprojektes von Stefanie Heidepriem, zu erforschen, wie ein textbasierter 
Chatbot zur Unterstützung psychisch kranker Menschen entwickelt werden kann.
Es existieren bereits Frameworks, die in der Lage sind Entitäten, also wichtige Schlüsselbegriffe, automatisiert aus
Texten zu extrahieren. Dies nennt man Named Entity Recognition.

Named Entity Recognition ... [TODO, einige Worte zu NER, aktuelle Entwicklungen, Erkenntnisse im medizinischen Bereich]

Das übergeordnete Dissertationsprojekt von Stefanie Heidepriem ist unter anderem an das Projekt MENHIR angelehnt. Dieses
Projekt dient der Erforschung von Konversationstechnologien, die psychisch kranke Menschen ünterstützen sollen. In diesem
Zusammenhang wird auch ein MENHIR-Chatbot entwickelt, der psychisch kranken Menschen personalisierte Unterstützung und 
hilfreiche Bewältigungsstrategien bieten soll.

