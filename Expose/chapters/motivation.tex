% !TEX root =expose.tex
\section{Motivation}

Die spezielle Herausforderung der Praktikumsaufgabe liegt darin, automatisiert Zusammenhänge zwischen Entitäten medizinischer Texte zu finden und diese Zusammenhänge zu repräsentieren. Dem zugrunde liegt das Ziel des Dissertationsprojektes von Stephanie Heidepriem, zu erforschen, wie ein textbasierter Chatbot zur Unterstützung 
psychisch kranker Menschen entwickelt werden kann. 

Das Dissertationsprojekt  ist unter anderem an das Projekt \emph{MENHIR} angelehnt \cite{menhir}. Dieses Projekt dient der Erforschung von Konversationstechnologien, die psychisch kranke Menschen unterstützen sollen. In diesem Zusammenhang wird auch 
ein MENHIR-Chatbot entwickelt, der personalisierte Unterstützung und hilfreiche Bewältigungsstrategien bieten soll. 

Das Forschungsprojekt \emph{STop Obesity Platform} beschäftigt sich mit der Extraktion von Wissen unter anderem aus Chatbots, das anschließend aufbereitet und mit weiteren Informationen kombiniert werden soll. Dieses Wissen soll medizinischem Fachpersonal zur Verfügung gestellt werden und außerdem Menschen mit Adipositas dabei helfen, eine gesündere Ernährung einzuhalten \cite{stopobesity}.


Es ist aber auch denkbar, dass die Problemstellung der Praktikumsaufgabe auch darüber hinaus für weitere Anwendungen interessant ist und mögliche Ergebnisse in verschiedenen Gebieten eingesetzt werden können, in denen Informationen in (medizinischen) Texten zueinander in Zusammenhang gebracht werden sollen.

Es wurde bereits viel an Methoden gearbeitet, die in der Lage sind Entitäten, also wichtige Objekte wie zum Beispiel Personen, 
Organisationen oder Orte, automatisiert aus Texten zu extrahieren. Dies nennt man Named Entity Recognition. Named Entity 
Recognition ist ein Bereich des Natural Language Processing und seit circa 30 Jahren werden unterschiedliche Techniken 
zum Lösen der Aufgaben, die in diesen Bereich fallen, entwickelt [Quelle: Recent Trends NER]. Darunter fallen 
grammatikbasierte und statistische Methoden wie auch Methoden des Machine Learning.

Außerdem aktualisiert die "United States National Library of Medicine" regelmäßig ein auf den medizinischen Bereich 
spezialisiertes Vokabular, beziehungsweise eine Wissensbasis, MeSH \cite{mesh}
Die "United States National Library of Medicine" hat unter anderem auch schon an einem Named-Entity Recognizer,
"MetaMapLite", gearbeitet und stellt diesen zur Verfügung. \cite{metamaplite}

