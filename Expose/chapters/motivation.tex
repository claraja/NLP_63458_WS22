% !TEX root =expose.tex
\chapter{Motivation}

Die spezielle Herausforderung der Praktikumsaufgabe liegt darin, automatisiert Zusammenhänge zwischen Entitäten 
medizinischer Texte zu finden und diese Zusammenhänge zu repräsentieren. Dem zugrunde liegt das Ziel des 
Dissertationsprojektes von Stefanie Heidepriem, zu erforschen, wie ein textbasierter Chatbot zur Unterstützung 
psychisch kranker Menschen entwickelt werden kann. 
Das Dissertationsprojekt  ist unter anderem an das Projekt \glqq MENHIR\grqq angelehnt \cite{menhir}. Dieses Projekt dient der Erforschung 
von Konversationstechnologien, die psychisch kranke Menschen unterstützen sollen. In diesem Zusammenhang wird auch 
ein MENHIR-Chatbot entwickelt, der personalisierte Unterstützung und hilfreiche Bewältigungsstrategien bieten soll. 
Das Forschungsprojekt \glqq STop Obesity Platform\grqq beschäftigt sich mit der Extraktion von Wissen unter anderem aus Chatbots, 
das anschließend aufbereitet und mit weiteren Informatinen kombiniert werden soll. Dieses Wissen soll anschließend
medizinischem Fachpersonal zur Verfügung gestellt werden um Menschen mit Übergewicht zu einer gesünderen Lebensweise
zu animieren \cite{stopobesity}.
Es ist aber auch denkbar, dass die Problemstellung der Praktikumsaufgabe auch darüber hinaus für weitere Anwendungen 
interessant sein könnte und mögliche Ergebnisse in verschiedenen Gebieten eingesetzt werden könnten, in denen 
Informationen in (medizinischen) Texten zueinander in Zusammenhang gebracht werden sollen.

Es existieren bereits Methoden, die in der Lage sind Entitäten, also wichtige Objekte wie zum Beispiel Personen, 
Organisationen oder Orte, automatisiert aus Texten zu extrahieren. Dies nennt man Named Entity Recognition. Named Entity 
Recognition ist ein Bereich des Natural Language Processing und seit circa 30 Jahren werden unterschiedliche Techniken 
zum Lösen der Aufgaben, die in diesen Bereich fallen, entwickelt \cite{trends_in_ner}. Darunter fallen 
grammatikbasierte und statistische Methoden wie auch Methoden des Machine Learning.

Außerdem exitistiert schon ein auf den medizinischen Bereich spezialisiertes Vokabular, beziehungsweise eine Wissensbasis, 
MeSH \cite{mesh}, das regelmäßig aktualisiert wird. Dieses Vokabular wurde von 
der "United States National Library of Medicine" erstellt. Diese Bibliothek stellt auch schon den Named-Entity Recognizer 
"MetaMapLite" zur Verfügung, der auch an die Zwecke der Nutzer angepasst und erweitert werden kann \cite{metamaplite}. 

