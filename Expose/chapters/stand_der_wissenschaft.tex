\chapter{Stand der Wissenschaft}
\label{ch:standderwissenschaft} 


%
% Section: FZ 1.1. Aufbau und Struktur medizinischer Texte 
%
\section{FZ 1.1. Aufbau und Struktur medizinischer Texte}
\label{sec:fz1.1.} 

Medizinische Texte existieren in verschiedenen Formen mit unterschiedlichen Zielsetzungen. Beispielsweise gibt es Lehrbuch- und Enzyklopädietexte, die überblicksweise über medizinische Sachverhalte informieren, wissenschaftliche Studien, in denen neue wissenschaftliche Erkenntnisse vorgestellt werden und außerdem Arztberichte, in denen über individuelle Patienten und deren Krankheitsverlauf informiert wird.
Die für diese Arbeit genutzten Analysetexte stellen enzyklopädische Texte dar, in denen insbesondere Symptome und Merkmale psychologischer Erkrankungen beschrieben werden.
Charakterisiert sind diese Texte dadurch, dass sie durch Zwischenüberschriften gegliedert sind und neben Fließtext auch Aufzählungen enthalten.


%
% Section: FZ 2.1. Überblick über depressive Erkrankungen 
%
\section{FZ 2.1. Überblick über depressive Erkrankungen}
\label{sec:fz2.1.} 


%
% Section: FZ 2.2. Automatisierung durch NLP 
%
\section{FZ 2.2. Automatisierung durch NLP}
\label{sec:fz2.2.} 


%
% Section: FZ 1.1. Wissensrepräsentation mittels RDF
%
\section{FZ 3.1. Wissensrepräsentation mittels RDF}
\label{sec:fz3.1.} 

Das \emph{Resource Description Framework} (RDF) \citep{w3c_all_2022} ist ein Framework zur Darstellung von Informationen im Semantischen Web, das von der \emph{RDF Working Group} des  \emph{World Wide Web Consortium} (W3C) erstellt wurde. Das RDF-Modell besteht aus einem Datenmodell, mit dem Aussagen über Ressourcen in Form eines Graphen dargestellt werden. Die Informationen werden als Tripel von Subjekt, Prädikat und Objekt gespeichert und ermöglichen auf diese Weise eine maschinenlesbare Bereitstellung semantischer Informationen.


Für Python steht mit RDFLib \cite{rdflib_team_rdflib_2022} eine Bibliothek zur Arbeit mit RDF-Graphen bereit.
