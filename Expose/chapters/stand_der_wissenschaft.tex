\chapter{Stand der Wissenschaft}
\label{ch:standderwissenschaft} 


%
% Section: FZ 1.1. Aufbau und Struktur medizinischer Texte 
%
\section{FZ 1.1. Aufbau und Struktur medizinischer Texte}
\label{sec:fz1.1.} 

Medizinische Texte existieren in verschiedenen Formen mit unterschiedlichen Zielsetzungen. Beispielsweise gibt es Lehrbuch- und Enzyklopädietexte, die überblicksweise über medizinische Sachverhalte informieren, wissenschaftliche Studien, in denen neue wissenschaftliche Erkenntnisse vorgestellt werden und außerdem Arztberichte, in denen über individuelle Patienten und deren Krankheitsverlauf informiert wird.
Die für diese Arbeit genutzten Analysetexte stellen enzyklopädische Texte dar, in denen insbesondere Symptome und Merkmale psychologischer Erkrankungen beschrieben werden.
Charakterisiert sind diese Texte dadurch, dass sie durch Zwischenüberschriften gegliedert sind und neben Fließtext auch Aufzählungen enthalten.


%
% Section: FZ 2.1. Überblick über depressive Erkrankungen 
%
\section{FZ 2.1. Überblick über depressive Erkrankungen}
\label{sec:fz2.1.} 
Als Quellen für diesen Abschnitt dienten \cite{mpg_depression}, \cite{who_depression} und \cite{psychrembel_depression}.
\subsection{Allgemein}
Depression ist eine psychische Krankheit, die geschätzt $3,8 \%$ der Weltbevölkerung und $5\%$ der Erwachsenen betrifft.
Sie kann dazu führen, dass betroffene Menschen Schwierigkeiten haben im Arbeits- und Familienleben zurecht zu
kommen, und sie führt im schlimmsten Fall zur Selbsttötung.

\subsection{Symptome}
Übliche Symptome eine Depression sind eine traurige Grundstimmung, Konzentrationsprobleme, Hoffnungslosigkeit, 
Müdigkeit und Reizbarkeit. Auch Schlafstörungen, eine Libidostörung, verminderter oder gesteigerter Appetit und
ein vermindertes Selbswertgefühl oder Selbstbewusstsein sind Symptome. Im schlimmsten Fall treten 
Selbsttötungsgedanken auf.

\subsection{Formen der Depression}
Die Schwere einer typischen Depression wird durch leichte, mittelgradige und schwere Episoden unterschieden. Dabei 
sind die Übergänge fließend.
Besondere Formen der Depression sind die chronische Depression, bei der trotz therapeutischer Maßnahmen nur wenig 
Besserung eintritt. Desweiteren gibt es die manisch-depressive Depression, bei der sich depressive und manische 
Phasen abwechseln. Außerdem gibt es auch kurze, akute depressive Verstimmungen, die nur zwischen einem Tag und
zwei Wochen dauern. 

\subsection{Ursachen}
Depressionen entstehen durch ein kompexes Zusammenspiel von sozialen, psychologischen und biologischen Faktoren.
Dabei wird angenommen, dass für eine typische Depression die Genetik 50% der Ursachen ausmacht. 
Konkret können Kindheitserfahrungen, Verluste und Arbeitslosigkeit eine Depression begünstigen.

\subsection{Behandlung}
Je nach Schweregrad der Krankheit werden zur Behandlung von Depressionen psychologische Behandlungen angewandt, 
und/oder den betroffenen Personen Antidepressiva verschrieben.


%
% Section: FZ 2.2. Automatisierung durch NLP 
%
\section{FZ 2.2. Automatisierung durch NLP}
\label{sec:fz2.2.} 

\subsection{Natural Language Processing (NLP)}
Natural Language Processing umfasst \dots{} [TODO]

\subsection{Named Entity Recognition (NER)}
Named Entity Recognition bezeichnet einen Teilbereich des Natural Language Processing, bei dem es darum geht 
wichtige Entitäten, wie zum Beispiel Personen, Orte oder Institutionen, in einem Text zu erkennen. Methoden 
zur Bewältigung dieser Aufgabe werden seit circa 30 Jahren entwickelt. Darunter fallen grammatikbasierte und 
statistische Methoden, sowie auch Methoden des Machine Learning.

\subsection{Entity Linking}
Als Entity Linking wird im Bereich des NLP die Aufgabe beschrieben, die den Entitäten (z.B. Personen, Orte) 
in einem Text das korrekte Äquivalent in einer Wissensbasis zuordnet.
In \cite{shen_entity_2021} wird Entity Linking folgendermaßen definiert (übersetzt):
\begin{defn}[Entity Linking]
Gegeben sei ein Dokument $D$, welches die erkannten Entitäten $M=\{m_1, m_2, \dots, m_{|M|}\}$ enthält, sowie
eine Ziel-Wissensbasis $KB$, welches die Entitäten $E=\{e_1, e_2, \dots, e_{|M|}\}$ enthält. Das Ziel ist es 
jede Entität $m_i$ in $M$ seinem korrekten Äquivalent $e_i$ in $E$ zuzuordnen.
\end{defn}

\subsection{spaCy}
spaCy \dots{} [TODO]

\subsection{Weitere Python-Bibliotheken für Machine Learning (ML) und NER}
Eine der wichtigsten Python-Bibliotheken für NLP ist NLTK (Natural Language Toolkit) \cite{bird2006nltk}.
Weitere Python-Bibliotheken, die für NLP-Aufgaben genutzt werden können, sind unter Anderem
Gensim \cite{vrehuuvrek2011gensim}, Pattern \cite{de2012pattern}\cite{github_pattern}, 
scikit-learn und PyTorch.

%
% Section: FZ 1.1. Wissensrepräsentation mittels RDF
%
\section{FZ 3.1. Wissensrepräsentation mittels RDF}
\label{sec:fz3.1.} 

Das \emph{Resource Description Framework} (RDF) \citep{w3c_all_2022} ist ein Framework zur Darstellung von Informationen im Semantischen Web, das von der \emph{RDF Working Group} des  \emph{World Wide Web Consortium} (W3C) erstellt wurde. Das RDF-Modell besteht aus einem Datenmodell, mit dem Aussagen über Ressourcen in Form eines Graphen dargestellt werden. Die Informationen werden als Tripel von Subjekt, Prädikat und Objekt gespeichert und ermöglichen auf diese Weise eine maschinenlesbare Bereitstellung semantischer Informationen.


Für Python steht mit RDFLib \cite{rdflib_team_rdflib_2022} eine Bibliothek zur Arbeit mit RDF-Graphen bereit.
