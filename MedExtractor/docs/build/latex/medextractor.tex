%% Generated by Sphinx.
\def\sphinxdocclass{report}
\documentclass[letterpaper,10pt,english]{sphinxmanual}
\ifdefined\pdfpxdimen
   \let\sphinxpxdimen\pdfpxdimen\else\newdimen\sphinxpxdimen
\fi \sphinxpxdimen=.75bp\relax
\ifdefined\pdfimageresolution
    \pdfimageresolution= \numexpr \dimexpr1in\relax/\sphinxpxdimen\relax
\fi
%% let collapsible pdf bookmarks panel have high depth per default
\PassOptionsToPackage{bookmarksdepth=5}{hyperref}

\PassOptionsToPackage{warn}{textcomp}
\usepackage[utf8]{inputenc}
\ifdefined\DeclareUnicodeCharacter
% support both utf8 and utf8x syntaxes
  \ifdefined\DeclareUnicodeCharacterAsOptional
    \def\sphinxDUC#1{\DeclareUnicodeCharacter{"#1}}
  \else
    \let\sphinxDUC\DeclareUnicodeCharacter
  \fi
  \sphinxDUC{00A0}{\nobreakspace}
  \sphinxDUC{2500}{\sphinxunichar{2500}}
  \sphinxDUC{2502}{\sphinxunichar{2502}}
  \sphinxDUC{2514}{\sphinxunichar{2514}}
  \sphinxDUC{251C}{\sphinxunichar{251C}}
  \sphinxDUC{2572}{\textbackslash}
\fi
\usepackage{cmap}
\usepackage[T1]{fontenc}
\usepackage{amsmath,amssymb,amstext}
\usepackage{babel}



\usepackage{tgtermes}
\usepackage{tgheros}
\renewcommand{\ttdefault}{txtt}



\usepackage[Bjarne]{fncychap}
\usepackage{sphinx}

\fvset{fontsize=auto}
\usepackage{geometry}


% Include hyperref last.
\usepackage{hyperref}
% Fix anchor placement for figures with captions.
\usepackage{hypcap}% it must be loaded after hyperref.
% Set up styles of URL: it should be placed after hyperref.
\urlstyle{same}

\addto\captionsenglish{\renewcommand{\contentsname}{Contents:}}

\usepackage{sphinxmessages}
\setcounter{tocdepth}{1}



\title{MedExtractor}
\date{Mar 16, 2023}
\release{1.0.0}
\author{Fapra Gruppe 5}
\newcommand{\sphinxlogo}{\vbox{}}
\renewcommand{\releasename}{Release}
\makeindex
\begin{document}

\ifdefined\shorthandoff
  \ifnum\catcode`\=\string=\active\shorthandoff{=}\fi
  \ifnum\catcode`\"=\active\shorthandoff{"}\fi
\fi

\pagestyle{empty}
\sphinxmaketitle
\pagestyle{plain}
\sphinxtableofcontents
\pagestyle{normal}
\phantomsection\label{\detokenize{index::doc}}

\index{module@\spxentry{module}!medextractor.medextractor@\spxentry{medextractor.medextractor}}\index{medextractor.medextractor@\spxentry{medextractor.medextractor}!module@\spxentry{module}}
\sphinxstepscope


\chapter{Fachpraktikum WS 22/23 \sphinxhyphen{} Natural Language Processing (NLP) mit spaCy}
\label{\detokenize{readme:fachpraktikum-ws-22-23-natural-language-processing-nlp-mit-spacy}}\label{\detokenize{readme::doc}}

\chapter{Medextractor \sphinxhyphen{} Konsolenapplikation}
\label{\detokenize{readme:medextractor-konsolenapplikation}}
\sphinxAtStartPar
Die Medextractor\sphinxhyphen{}Konsolenapplikation analysiert Texte und sucht darin
nach Krankheiten und deren Symptomen und erstellt eine
Wissensrepräsentation, die die gefundenen Krankheiten und Symptome
miteinander in Beziehung setzt. Die Wissensrepräsentation wird im RDF
(Resource Discription Framework)\sphinxhyphen{}Format gespeichert. Zusätzlich erstellt
der Medextractor eine xml\sphinxhyphen{}Datei mit Daten für den Entity Linker von
spaCy sowie eine Datei name.kb, in dem die erstellte Wissensbasis in
Binärcode abgespeichert wird.


\chapter{Ordnerstruktur}
\label{\detokenize{readme:ordnerstruktur}}\begin{itemize}
\item {} 
\sphinxAtStartPar
\sphinxstylestrong{docs}: mit Sphinx erstellte Dokumentation des
MedExtractor\sphinxhyphen{}Programms, mit Öffnen der html\sphinxhyphen{}Dateien in docs/build
kommt man zu einer ansprechenden Dokumentation des Programms

\item {} 
\sphinxAtStartPar
\sphinxstylestrong{medextractor}: enthält die Programme, die für die Erstellung der
Wissensbasis zuständig sind
\begin{itemize}
\item {} 
\sphinxAtStartPar
\sphinxstylestrong{preprocessor}: Programm, das einen gegebenen Text
vorverarbeitet, sodass es in weiteren Schritten besser verarbeitet
werden kann

\item {} 
\sphinxAtStartPar
\sphinxstylestrong{knowledge}: verarbeitet vorverarbeiteten Text, extrahiert
Krankheiten und dazugehörige Symptome und speichert sie in einer
Knowledgebase

\item {} 
\sphinxAtStartPar
\sphinxstylestrong{rdf}: Serialisiert die erstellte Knowledgebase und speichert
sie als xml\sphinxhyphen{}Datei im RDF\sphinxhyphen{}Format

\end{itemize}

\item {} 
\sphinxAtStartPar
\sphinxstylestrong{resources}: alle Ressourcen, die zum Ausführen des Programms
benötigt werde, sowie Ausgabedateien
\begin{itemize}
\item {} 
\sphinxAtStartPar
\sphinxstylestrong{to\_analyze}: Texte, die vom Programm analysiert und ausgewertet
werden können

\item {} 
\sphinxAtStartPar
\sphinxstylestrong{training\_data}: Vokabular\sphinxhyphen{}Dateien (siehe Abschnitt unten)

\end{itemize}

\item {} 
\sphinxAtStartPar
\sphinxstylestrong{test}: Testprogramme

\end{itemize}


\chapter{Konfigurationsdatei config.json}
\label{\detokenize{readme:konfigurationsdatei-config-json}}
\sphinxAtStartPar
Das Python\sphinxhyphen{}Modul, mit dem die Konsolenapplikation gestartet wird, ist
die Datei: medextractor.py. Im selben Order von medextractor.py muss
sich die Konfigurationsdatei config.json befinden.

\sphinxAtStartPar
Sollte es noch keine config.json Datei geben, wird beim Aufruf von
medextractor.py eine Beispiel\sphinxhyphen{}Datei erzeugt, die anschließend vom Nutzer
angepasst werden muss.

\sphinxAtStartPar
Die Konfigurationsdatei enthält folgende Informationen:
\begin{enumerate}
\sphinxsetlistlabels{\arabic}{enumi}{enumii}{}{.}%
\item {} 
\sphinxAtStartPar
Pfad und Name der xml\sphinxhyphen{}Datei für den Export im RDF\sphinxhyphen{}Format

\item {} 
\sphinxAtStartPar
Pfad und Name der xml\sphinxhyphen{}Datei für den Export für den Entity Linker

\item {} 
\sphinxAtStartPar
Pfad und Name der KnowledgeBase Datei

\item {} 
\sphinxAtStartPar
Pfad zu dem Order, der die zu analysierenden Texte enthält

\item {} 
\sphinxAtStartPar
Spezifikation, ob die Knowledgebase Datei überschrieben werden soll
(True oder False)

\item {} 
\sphinxAtStartPar
Pfad und Name der .txt\sphinxhyphen{}Datei, die das Krankheiten\sphinxhyphen{}Vokabular enthält

\item {} 
\sphinxAtStartPar
Pfad und Name der .txt\sphinxhyphen{}Datei, die das Symptome\sphinxhyphen{}Vokabular enthält

\end{enumerate}

\sphinxAtStartPar
Die Pfade müssen relativ zu dem Order angegeben werden, in dem sich
medextractor.py befindet. Alternativ können auch absolute Pfade angeben
werden.

\sphinxAtStartPar
Es werden alle Textdateien ({\color{red}\bfseries{}*}.txt) analysiert, die sich in dem in der
config.json\sphinxhyphen{}Datei angegebenen Ordner befinden. Die von Medextractor
erzeugten xml\sphinxhyphen{} und Knowledgebase\sphinxhyphen{} Dateien enthalten ein über alle
analysierten Texte akkumuliertes Ergebnis.

\sphinxAtStartPar
Wird festgelegt, dass die Knowledgebase\sphinxhyphen{}Datei nicht überschrieben werden
soll, werden alle neu gefundenen Krankheit\sphinxhyphen{}Symptom\sphinxhyphen{}Beziehungen zu der
vorhandenen Knowledgebase\sphinxhyphen{}Datei hinzugefügt.


\chapter{Vokabular\sphinxhyphen{}Dateien}
\label{\detokenize{readme:vokabular-dateien}}
\sphinxAtStartPar
Die Vokabulardateien sind einfache Dateien im csv\sphinxhyphen{}Format und enthalten
Einträge der folgenden Art:

\sphinxAtStartPar
C0010051 coronary aneurysm DISEASE

\sphinxAtStartPar
Der Eintrag C0010051 ist der CUI (Concept Unique Identifier) aus der
MetaMapLite\sphinxhyphen{}Datenbank. Der CUI ist als Referenz enthalten, wird aber
nicht weiter vom Medextractor verwendet.


\chapter{Aufruf des Programms}
\label{\detokenize{readme:aufruf-des-programms}}

\section{Voraussetzungen}
\label{\detokenize{readme:voraussetzungen}}\begin{itemize}
\item {} 
\sphinxAtStartPar
eine Python Version 3.6\sphinxhyphen{}3.8 (empfohlen und getestet: 3.8) muss
installiert sein (SpaCy ist noch nicht kompatibel mit Python \textgreater{}3.8)

\item {} 
\sphinxAtStartPar
Packages, die in requirements.txt aufgelistet sind, sind installiert
(Installation aller Packages möglich mit dem Befehl
\sphinxcode{\sphinxupquote{pip install \sphinxhyphen{}r requirements.txt}})

\end{itemize}


\section{Aufruf}
\label{\detokenize{readme:aufruf}}
\sphinxAtStartPar
Das Programm wird gestartet, indem in die Windowseingabeaufforderung der
Befehl

\sphinxAtStartPar
\sphinxcode{\sphinxupquote{python medextractor.py}}

\sphinxAtStartPar
eingegeben wird.

\sphinxAtStartPar
Zu beachten ist, dass in der System\sphinxhyphen{}Path\sphinxhyphen{}Umgebungsvariable der Pfad zur
(ggf. virtuellen) Umgebung des Python\sphinxhyphen{}Interpreters enthalten ist, in der
spaCy installiert wurde. Ggf. sollte hierzu activate.bat im Verzeichnis
der virtuellen Umgebung der Python\sphinxhyphen{}Installation aufgerufen werden.

\sphinxAtStartPar
Da die Vokabulardateien umfangreich sind, kann allein das Trainieren des
Entity\sphinxhyphen{}Rulers (je nach Rechner) eine Minute übersteigen.

\sphinxAtStartPar
Nach Beendigung des Programms befinden sich die xml\sphinxhyphen{}Dateien mit der
RDF\sphinxhyphen{}Repräsentation sowie die xml\sphinxhyphen{}Datei für den Entity Linker in dem in
config.json angegebenen Ordner.


\chapter{Entity\sphinxhyphen{}Linker}
\label{\detokenize{readme:entity-linker}}
\sphinxAtStartPar
Das Jupyter\sphinxhyphen{}Notebook entity\_linker\_demo.ipynb (zu finden im Ordner
NLP\_63458\_WS22/notebooks/entity\_linker\_demo.ipynb) demonstriert, wie die
Daten aus der xml\sphinxhyphen{}Export\sphinxhyphen{}Datei gelesen und für das Training von Entity
Ruler und Entity Linker verwendet werden. Findet der Entity Ruler in
einem Text Symptome, dann ordnet der Entity Linker diesen Symptome
dazugehörige Krankheiten zu.

\sphinxstepscope


\chapter{medextractor}
\label{\detokenize{modules:medextractor}}\label{\detokenize{modules::doc}}
\sphinxstepscope


\section{medextractor package}
\label{\detokenize{medextractor:medextractor-package}}\label{\detokenize{medextractor::doc}}

\subsection{Subpackages}
\label{\detokenize{medextractor:subpackages}}
\sphinxstepscope


\subsubsection{medextractor.knowledge package}
\label{\detokenize{medextractor.knowledge:medextractor-knowledge-package}}\label{\detokenize{medextractor.knowledge::doc}}

\paragraph{Submodules}
\label{\detokenize{medextractor.knowledge:submodules}}

\paragraph{medextractor.knowledge.base module}
\label{\detokenize{medextractor.knowledge:module-medextractor.knowledge.base}}\label{\detokenize{medextractor.knowledge:medextractor-knowledge-base-module}}\index{module@\spxentry{module}!medextractor.knowledge.base@\spxentry{medextractor.knowledge.base}}\index{medextractor.knowledge.base@\spxentry{medextractor.knowledge.base}!module@\spxentry{module}}\index{KnowledgeBase (class in medextractor.knowledge.base)@\spxentry{KnowledgeBase}\spxextra{class in medextractor.knowledge.base}}

\begin{fulllineitems}
\phantomsection\label{\detokenize{medextractor.knowledge:medextractor.knowledge.base.KnowledgeBase}}
\pysigstartsignatures
\pysigline{\sphinxbfcode{\sphinxupquote{class\DUrole{w}{  }}}\sphinxcode{\sphinxupquote{medextractor.knowledge.base.}}\sphinxbfcode{\sphinxupquote{KnowledgeBase}}}
\pysigstopsignatures
\sphinxAtStartPar
Bases: \sphinxcode{\sphinxupquote{object}}

\sphinxAtStartPar
The KnowledgeBase manages entities and relations.

\sphinxAtStartPar
Functions:
add\_relation(SemanticRelation)
has\_relation(SemanticRelation) \sphinxhyphen{}\textgreater{} bool
give\_entities(str) \sphinxhyphen{}\textgreater{} {[}{]}
export\_for\_entity\_linker(str)
safe(str)
load(str)
\index{add\_relation() (medextractor.knowledge.base.KnowledgeBase method)@\spxentry{add\_relation()}\spxextra{medextractor.knowledge.base.KnowledgeBase method}}

\begin{fulllineitems}
\phantomsection\label{\detokenize{medextractor.knowledge:medextractor.knowledge.base.KnowledgeBase.add_relation}}
\pysigstartsignatures
\pysiglinewithargsret{\sphinxbfcode{\sphinxupquote{add\_relation}}}{\emph{\DUrole{n}{relation}\DUrole{p}{:}\DUrole{w}{  }\DUrole{n}{{\hyperref[\detokenize{medextractor.knowledge:medextractor.knowledge.semantics.SemanticRelation}]{\sphinxcrossref{SemanticRelation}}}}}}{{ $\rightarrow$ None}}
\pysigstopsignatures
\sphinxAtStartPar
Add a SemanticRelation into the KnowledgeBase.
\begin{quote}\begin{description}
\sphinxlineitem{Parameters}
\sphinxAtStartPar
\sphinxstyleliteralstrong{\sphinxupquote{relation}} ({\hyperref[\detokenize{medextractor.knowledge:medextractor.knowledge.semantics.SemanticRelation}]{\sphinxcrossref{\sphinxstyleliteralemphasis{\sphinxupquote{SemanticRelation}}}}}) \textendash{} 

\end{description}\end{quote}

\end{fulllineitems}

\index{add\_training\_example\_to\_relation() (medextractor.knowledge.base.KnowledgeBase method)@\spxentry{add\_training\_example\_to\_relation()}\spxextra{medextractor.knowledge.base.KnowledgeBase method}}

\begin{fulllineitems}
\phantomsection\label{\detokenize{medextractor.knowledge:medextractor.knowledge.base.KnowledgeBase.add_training_example_to_relation}}
\pysigstartsignatures
\pysiglinewithargsret{\sphinxbfcode{\sphinxupquote{add\_training\_example\_to\_relation}}}{\emph{\DUrole{n}{relation}\DUrole{p}{:}\DUrole{w}{  }\DUrole{n}{{\hyperref[\detokenize{medextractor.knowledge:medextractor.knowledge.semantics.SemanticRelation}]{\sphinxcrossref{SemanticRelation}}}}}, \emph{\DUrole{n}{sent\_text}\DUrole{p}{:}\DUrole{w}{  }\DUrole{n}{str}}}{{ $\rightarrow$ None}}
\pysigstopsignatures
\sphinxAtStartPar
Add a training sentence to a SemanticRelation
\begin{quote}\begin{description}
\sphinxlineitem{Parameters}\begin{itemize}
\item {} 
\sphinxAtStartPar
\sphinxstyleliteralstrong{\sphinxupquote{relation}} ({\hyperref[\detokenize{medextractor.knowledge:medextractor.knowledge.semantics.SemanticRelation}]{\sphinxcrossref{\sphinxstyleliteralemphasis{\sphinxupquote{SemanticRelation}}}}}) \textendash{} 

\item {} 
\sphinxAtStartPar
\sphinxstyleliteralstrong{\sphinxupquote{sent\_text}} (\sphinxstyleliteralemphasis{\sphinxupquote{str}}) \textendash{} training sentence

\end{itemize}

\end{description}\end{quote}

\end{fulllineitems}

\index{export\_for\_entity\_linker() (medextractor.knowledge.base.KnowledgeBase method)@\spxentry{export\_for\_entity\_linker()}\spxextra{medextractor.knowledge.base.KnowledgeBase method}}

\begin{fulllineitems}
\phantomsection\label{\detokenize{medextractor.knowledge:medextractor.knowledge.base.KnowledgeBase.export_for_entity_linker}}
\pysigstartsignatures
\pysiglinewithargsret{\sphinxbfcode{\sphinxupquote{export\_for\_entity\_linker}}}{\emph{\DUrole{n}{file\_name}\DUrole{p}{:}\DUrole{w}{  }\DUrole{n}{str}}}{}
\pysigstopsignatures
\sphinxAtStartPar
Save the KnowledgeBase data as an xml file that can be used to train an entity linker.
\begin{quote}\begin{description}
\sphinxlineitem{Parameters}
\sphinxAtStartPar
\sphinxstyleliteralstrong{\sphinxupquote{file\_name}} (\sphinxstyleliteralemphasis{\sphinxupquote{str}}) \textendash{} the name of the xml file

\end{description}\end{quote}

\end{fulllineitems}

\index{get\_entities() (medextractor.knowledge.base.KnowledgeBase method)@\spxentry{get\_entities()}\spxextra{medextractor.knowledge.base.KnowledgeBase method}}

\begin{fulllineitems}
\phantomsection\label{\detokenize{medextractor.knowledge:medextractor.knowledge.base.KnowledgeBase.get_entities}}
\pysigstartsignatures
\pysiglinewithargsret{\sphinxbfcode{\sphinxupquote{get\_entities}}}{\emph{\DUrole{n}{alias}\DUrole{p}{:}\DUrole{w}{  }\DUrole{n}{str}}}{{ $\rightarrow$ \DUrole{p}{{[}}\DUrole{p}{{]}}}}
\pysigstopsignatures
\sphinxAtStartPar
Return a list of entities that are related to symptoms in SemanticRelations stored in the KnowledgeBase
\begin{quote}\begin{description}
\sphinxlineitem{Parameters}
\sphinxAtStartPar
\sphinxstyleliteralstrong{\sphinxupquote{alias}} (\sphinxstyleliteralemphasis{\sphinxupquote{str}}) \textendash{} the name of a symptom

\sphinxlineitem{Return type}
\sphinxAtStartPar
list of Entity

\end{description}\end{quote}

\end{fulllineitems}

\index{has\_relation() (medextractor.knowledge.base.KnowledgeBase method)@\spxentry{has\_relation()}\spxextra{medextractor.knowledge.base.KnowledgeBase method}}

\begin{fulllineitems}
\phantomsection\label{\detokenize{medextractor.knowledge:medextractor.knowledge.base.KnowledgeBase.has_relation}}
\pysigstartsignatures
\pysiglinewithargsret{\sphinxbfcode{\sphinxupquote{has\_relation}}}{\emph{\DUrole{n}{relation}\DUrole{p}{:}\DUrole{w}{  }\DUrole{n}{{\hyperref[\detokenize{medextractor.knowledge:medextractor.knowledge.semantics.SemanticRelation}]{\sphinxcrossref{SemanticRelation}}}}}}{{ $\rightarrow$ bool}}
\pysigstopsignatures
\sphinxAtStartPar
Return True if the relation is in the KnowledgeBase. Return False otherwise.
\begin{quote}\begin{description}
\sphinxlineitem{Parameters}
\sphinxAtStartPar
\sphinxstyleliteralstrong{\sphinxupquote{relation}} ({\hyperref[\detokenize{medextractor.knowledge:medextractor.knowledge.semantics.SemanticRelation}]{\sphinxcrossref{\sphinxstyleliteralemphasis{\sphinxupquote{SemanticRelation}}}}}) \textendash{} 

\end{description}\end{quote}

\end{fulllineitems}

\index{load() (medextractor.knowledge.base.KnowledgeBase method)@\spxentry{load()}\spxextra{medextractor.knowledge.base.KnowledgeBase method}}

\begin{fulllineitems}
\phantomsection\label{\detokenize{medextractor.knowledge:medextractor.knowledge.base.KnowledgeBase.load}}
\pysigstartsignatures
\pysiglinewithargsret{\sphinxbfcode{\sphinxupquote{load}}}{\emph{\DUrole{n}{file\_name}\DUrole{p}{:}\DUrole{w}{  }\DUrole{n}{str}}}{}
\pysigstopsignatures
\sphinxAtStartPar
Load the KnowledgeBase from file.
\begin{quote}\begin{description}
\sphinxlineitem{Parameters}
\sphinxAtStartPar
\sphinxstyleliteralstrong{\sphinxupquote{file\_name}} (\sphinxstyleliteralemphasis{\sphinxupquote{str}}) \textendash{} the name of the file

\end{description}\end{quote}

\end{fulllineitems}

\index{save() (medextractor.knowledge.base.KnowledgeBase method)@\spxentry{save()}\spxextra{medextractor.knowledge.base.KnowledgeBase method}}

\begin{fulllineitems}
\phantomsection\label{\detokenize{medextractor.knowledge:medextractor.knowledge.base.KnowledgeBase.save}}
\pysigstartsignatures
\pysiglinewithargsret{\sphinxbfcode{\sphinxupquote{save}}}{\emph{\DUrole{n}{file\_name}\DUrole{p}{:}\DUrole{w}{  }\DUrole{n}{str}}}{{ $\rightarrow$ None}}
\pysigstopsignatures
\sphinxAtStartPar
Save the KnowledgeBase into a pickle file.
If the KnowledgeBase does not contain any SemanticRelations, no file is saved.
\begin{quote}\begin{description}
\sphinxlineitem{Parameters}
\sphinxAtStartPar
\sphinxstyleliteralstrong{\sphinxupquote{file\_name}} (\sphinxstyleliteralemphasis{\sphinxupquote{str}}) \textendash{} the name of the file

\end{description}\end{quote}

\end{fulllineitems}


\end{fulllineitems}



\paragraph{medextractor.knowledge.entity module}
\label{\detokenize{medextractor.knowledge:module-medextractor.knowledge.entity}}\label{\detokenize{medextractor.knowledge:medextractor-knowledge-entity-module}}\index{module@\spxentry{module}!medextractor.knowledge.entity@\spxentry{medextractor.knowledge.entity}}\index{medextractor.knowledge.entity@\spxentry{medextractor.knowledge.entity}!module@\spxentry{module}}\index{Entity (class in medextractor.knowledge.entity)@\spxentry{Entity}\spxextra{class in medextractor.knowledge.entity}}

\begin{fulllineitems}
\phantomsection\label{\detokenize{medextractor.knowledge:medextractor.knowledge.entity.Entity}}
\pysigstartsignatures
\pysiglinewithargsret{\sphinxbfcode{\sphinxupquote{class\DUrole{w}{  }}}\sphinxcode{\sphinxupquote{medextractor.knowledge.entity.}}\sphinxbfcode{\sphinxupquote{Entity}}}{\emph{\DUrole{n}{entity\_name}\DUrole{p}{:}\DUrole{w}{  }\DUrole{n}{str}}, \emph{\DUrole{n}{entity\_type}\DUrole{p}{:}\DUrole{w}{  }\DUrole{n}{{\hyperref[\detokenize{medextractor.knowledge:medextractor.knowledge.entity.EntityType}]{\sphinxcrossref{EntityType}}}}}}{}
\pysigstopsignatures
\sphinxAtStartPar
Bases: \sphinxcode{\sphinxupquote{object}}

\sphinxAtStartPar
An entity consisting of its name string and its EntityType

\end{fulllineitems}

\index{EntityType (class in medextractor.knowledge.entity)@\spxentry{EntityType}\spxextra{class in medextractor.knowledge.entity}}

\begin{fulllineitems}
\phantomsection\label{\detokenize{medextractor.knowledge:medextractor.knowledge.entity.EntityType}}
\pysigstartsignatures
\pysiglinewithargsret{\sphinxbfcode{\sphinxupquote{class\DUrole{w}{  }}}\sphinxcode{\sphinxupquote{medextractor.knowledge.entity.}}\sphinxbfcode{\sphinxupquote{EntityType}}}{\emph{\DUrole{n}{value}}}{}
\pysigstopsignatures
\sphinxAtStartPar
Bases: \sphinxcode{\sphinxupquote{Enum}}

\sphinxAtStartPar
Types of entities that can be stored in the KnowledgeBase.
\index{DISEASE (medextractor.knowledge.entity.EntityType attribute)@\spxentry{DISEASE}\spxextra{medextractor.knowledge.entity.EntityType attribute}}

\begin{fulllineitems}
\phantomsection\label{\detokenize{medextractor.knowledge:medextractor.knowledge.entity.EntityType.DISEASE}}
\pysigstartsignatures
\pysigline{\sphinxbfcode{\sphinxupquote{DISEASE}}\sphinxbfcode{\sphinxupquote{\DUrole{w}{  }\DUrole{p}{=}\DUrole{w}{  }1}}}
\pysigstopsignatures
\end{fulllineitems}

\index{SYMPTOM (medextractor.knowledge.entity.EntityType attribute)@\spxentry{SYMPTOM}\spxextra{medextractor.knowledge.entity.EntityType attribute}}

\begin{fulllineitems}
\phantomsection\label{\detokenize{medextractor.knowledge:medextractor.knowledge.entity.EntityType.SYMPTOM}}
\pysigstartsignatures
\pysigline{\sphinxbfcode{\sphinxupquote{SYMPTOM}}\sphinxbfcode{\sphinxupquote{\DUrole{w}{  }\DUrole{p}{=}\DUrole{w}{  }2}}}
\pysigstopsignatures
\end{fulllineitems}

\index{UNDEFINED (medextractor.knowledge.entity.EntityType attribute)@\spxentry{UNDEFINED}\spxextra{medextractor.knowledge.entity.EntityType attribute}}

\begin{fulllineitems}
\phantomsection\label{\detokenize{medextractor.knowledge:medextractor.knowledge.entity.EntityType.UNDEFINED}}
\pysigstartsignatures
\pysigline{\sphinxbfcode{\sphinxupquote{UNDEFINED}}\sphinxbfcode{\sphinxupquote{\DUrole{w}{  }\DUrole{p}{=}\DUrole{w}{  }3}}}
\pysigstopsignatures
\end{fulllineitems}


\end{fulllineitems}



\paragraph{medextractor.knowledge.knowledge\_extractor module}
\label{\detokenize{medextractor.knowledge:module-medextractor.knowledge.knowledge_extractor}}\label{\detokenize{medextractor.knowledge:medextractor-knowledge-knowledge-extractor-module}}\index{module@\spxentry{module}!medextractor.knowledge.knowledge\_extractor@\spxentry{medextractor.knowledge.knowledge\_extractor}}\index{medextractor.knowledge.knowledge\_extractor@\spxentry{medextractor.knowledge.knowledge\_extractor}!module@\spxentry{module}}\index{KnowledgeExtractor (class in medextractor.knowledge.knowledge\_extractor)@\spxentry{KnowledgeExtractor}\spxextra{class in medextractor.knowledge.knowledge\_extractor}}

\begin{fulllineitems}
\phantomsection\label{\detokenize{medextractor.knowledge:medextractor.knowledge.knowledge_extractor.KnowledgeExtractor}}
\pysigstartsignatures
\pysiglinewithargsret{\sphinxbfcode{\sphinxupquote{class\DUrole{w}{  }}}\sphinxcode{\sphinxupquote{medextractor.knowledge.knowledge\_extractor.}}\sphinxbfcode{\sphinxupquote{KnowledgeExtractor}}}{\emph{\DUrole{n}{config}\DUrole{p}{:}\DUrole{w}{  }\DUrole{n}{{\hyperref[\detokenize{medextractor:medextractor.config_manager.ConfigManager}]{\sphinxcrossref{ConfigManager}}}}}}{}
\pysigstopsignatures
\sphinxAtStartPar
Bases: \sphinxcode{\sphinxupquote{object}}

\sphinxAtStartPar
KnowledgeExtractor searches a text string for entities and for relations between these entities
\index{analyze\_linguistically() (medextractor.knowledge.knowledge\_extractor.KnowledgeExtractor method)@\spxentry{analyze\_linguistically()}\spxextra{medextractor.knowledge.knowledge\_extractor.KnowledgeExtractor method}}

\begin{fulllineitems}
\phantomsection\label{\detokenize{medextractor.knowledge:medextractor.knowledge.knowledge_extractor.KnowledgeExtractor.analyze_linguistically}}
\pysigstartsignatures
\pysiglinewithargsret{\sphinxbfcode{\sphinxupquote{analyze\_linguistically}}}{\emph{\DUrole{n}{text}}}{}
\pysigstopsignatures
\sphinxAtStartPar
Method that finds entities in a given text and outputs them on the command line
together with part\sphinxhyphen{}of\sphinxhyphen{}speech tags and the syntactic dependency within the sentence
\begin{quote}\begin{description}
\sphinxlineitem{Parameters}
\sphinxAtStartPar
\sphinxstyleliteralstrong{\sphinxupquote{text}} (\sphinxstyleliteralemphasis{\sphinxupquote{string}}) \textendash{} The text string to be analyzed by the method

\sphinxlineitem{Return type}
\sphinxAtStartPar
None

\end{description}\end{quote}

\end{fulllineitems}

\index{export\_for\_entity\_linker() (medextractor.knowledge.knowledge\_extractor.KnowledgeExtractor method)@\spxentry{export\_for\_entity\_linker()}\spxextra{medextractor.knowledge.knowledge\_extractor.KnowledgeExtractor method}}

\begin{fulllineitems}
\phantomsection\label{\detokenize{medextractor.knowledge:medextractor.knowledge.knowledge_extractor.KnowledgeExtractor.export_for_entity_linker}}
\pysigstartsignatures
\pysiglinewithargsret{\sphinxbfcode{\sphinxupquote{export\_for\_entity\_linker}}}{}{}
\pysigstopsignatures
\sphinxAtStartPar
Exports all entities, aliases and example sentences into an xml\sphinxhyphen{}File. The data
is prepared for easy import into spaCy’s Entity Linker. The xml\sphinxhyphen{}File is human
readable and allows reviewing the data that will be used by the Entity Linker.
Path and filename are defined in config.json.
\begin{quote}\begin{description}
\sphinxlineitem{Parameters}
\sphinxAtStartPar
\sphinxstyleliteralstrong{\sphinxupquote{None}} \textendash{} 

\sphinxlineitem{Return type}
\sphinxAtStartPar
None

\end{description}\end{quote}

\end{fulllineitems}

\index{get\_knowledge\_base() (medextractor.knowledge.knowledge\_extractor.KnowledgeExtractor method)@\spxentry{get\_knowledge\_base()}\spxextra{medextractor.knowledge.knowledge\_extractor.KnowledgeExtractor method}}

\begin{fulllineitems}
\phantomsection\label{\detokenize{medextractor.knowledge:medextractor.knowledge.knowledge_extractor.KnowledgeExtractor.get_knowledge_base}}
\pysigstartsignatures
\pysiglinewithargsret{\sphinxbfcode{\sphinxupquote{get\_knowledge\_base}}}{}{}
\pysigstopsignatures
\sphinxAtStartPar
Returns the knowledgebase that contains all entities and
sample sentences. Samples sentences can be used for training
statistical models (e.g. Entity Linker)
\begin{quote}\begin{description}
\sphinxlineitem{Parameters}
\sphinxAtStartPar
\sphinxstyleliteralstrong{\sphinxupquote{None}} \textendash{} 

\sphinxlineitem{Return type}
\sphinxAtStartPar
{\hyperref[\detokenize{medextractor.knowledge:medextractor.knowledge.base.KnowledgeBase}]{\sphinxcrossref{KnowledgeBase}}}

\end{description}\end{quote}

\end{fulllineitems}

\index{is\_related() (medextractor.knowledge.knowledge\_extractor.KnowledgeExtractor method)@\spxentry{is\_related()}\spxextra{medextractor.knowledge.knowledge\_extractor.KnowledgeExtractor method}}

\begin{fulllineitems}
\phantomsection\label{\detokenize{medextractor.knowledge:medextractor.knowledge.knowledge_extractor.KnowledgeExtractor.is_related}}
\pysigstartsignatures
\pysiglinewithargsret{\sphinxbfcode{\sphinxupquote{is\_related}}}{\emph{\DUrole{n}{entity1}}, \emph{\DUrole{n}{entity2}}, \emph{\DUrole{n}{sent}}}{}
\pysigstopsignatures
\sphinxAtStartPar
Returns relation type of entity1 and entity2. If both entities are
found to be unrelated, RelationType.NO\_RELATION is returned.

\sphinxAtStartPar
Parameter sent is not used because this function currently only implements a
very simple relation check without analyzing the syntax of the sentence. Such
analysis could be added at a later stage.

\sphinxAtStartPar
At the moment is\_related() just checks whether entity1 is a disease and whether
entity2 is a symptom. Thus possible results are only RelationType.NO\_RELATION
and RelationType.HAS\_SYMPTOM.
\begin{quote}\begin{description}
\sphinxlineitem{Parameters}\begin{itemize}
\item {} 
\sphinxAtStartPar
\sphinxstyleliteralstrong{\sphinxupquote{entity1}} (\sphinxstyleliteralemphasis{\sphinxupquote{spacy.Span}}) \textendash{} 

\item {} 
\sphinxAtStartPar
\sphinxstyleliteralstrong{\sphinxupquote{entity2}} (\sphinxstyleliteralemphasis{\sphinxupquote{spacy.Span}}) \textendash{} 

\item {} 
\sphinxAtStartPar
\sphinxstyleliteralstrong{\sphinxupquote{sent}} (\sphinxstyleliteralemphasis{\sphinxupquote{spacy.Span}}) \textendash{} 

\end{itemize}

\sphinxlineitem{Return type}
\sphinxAtStartPar
{\hyperref[\detokenize{medextractor.knowledge:medextractor.knowledge.relations.RelationType}]{\sphinxcrossref{RelationType}}} (Enum)

\end{description}\end{quote}

\end{fulllineitems}

\index{process\_texts() (medextractor.knowledge.knowledge\_extractor.KnowledgeExtractor method)@\spxentry{process\_texts()}\spxextra{medextractor.knowledge.knowledge\_extractor.KnowledgeExtractor method}}

\begin{fulllineitems}
\phantomsection\label{\detokenize{medextractor.knowledge:medextractor.knowledge.knowledge_extractor.KnowledgeExtractor.process_texts}}
\pysigstartsignatures
\pysiglinewithargsret{\sphinxbfcode{\sphinxupquote{process\_texts}}}{}{}
\pysigstopsignatures
\sphinxAtStartPar
Analyzes all text documents in the folder specified in config.json
\begin{quote}\begin{description}
\sphinxlineitem{Parameters}
\sphinxAtStartPar
\sphinxstyleliteralstrong{\sphinxupquote{None}} \textendash{} 

\sphinxlineitem{Return type}
\sphinxAtStartPar
None

\end{description}\end{quote}

\end{fulllineitems}

\index{saveKB() (medextractor.knowledge.knowledge\_extractor.KnowledgeExtractor method)@\spxentry{saveKB()}\spxextra{medextractor.knowledge.knowledge\_extractor.KnowledgeExtractor method}}

\begin{fulllineitems}
\phantomsection\label{\detokenize{medextractor.knowledge:medextractor.knowledge.knowledge_extractor.KnowledgeExtractor.saveKB}}
\pysigstartsignatures
\pysiglinewithargsret{\sphinxbfcode{\sphinxupquote{saveKB}}}{\emph{\DUrole{o}{*}\DUrole{n}{args}}}{}
\pysigstopsignatures
\sphinxAtStartPar
Saves the database persistently. Optionally, path and file name are given
as a string parameter when calling this function. If no path and file name
are given, the function will use the path and file name in attribute
self.\_config.knowledgebase\_filename.
\begin{quote}\begin{description}
\sphinxlineitem{Parameters}
\sphinxAtStartPar
\sphinxstyleliteralstrong{\sphinxupquote{file\_name}} (\sphinxstyleliteralemphasis{\sphinxupquote{string optional}}) \textendash{} 

\sphinxlineitem{Return type}
\sphinxAtStartPar
None

\end{description}\end{quote}

\end{fulllineitems}

\index{set\_context() (medextractor.knowledge.knowledge\_extractor.KnowledgeExtractor method)@\spxentry{set\_context()}\spxextra{medextractor.knowledge.knowledge\_extractor.KnowledgeExtractor method}}

\begin{fulllineitems}
\phantomsection\label{\detokenize{medextractor.knowledge:medextractor.knowledge.knowledge_extractor.KnowledgeExtractor.set_context}}
\pysigstartsignatures
\pysiglinewithargsret{\sphinxbfcode{\sphinxupquote{set\_context}}}{\emph{\DUrole{n}{context}}}{}
\pysigstopsignatures
\sphinxAtStartPar
This function allows defining a context. The context is described by
named entities included in the Entity Ruler (self.\_ruler). These entities will
be added to the set of entities when searching for disease/symptom relations
between entities.
\begin{quote}\begin{description}
\sphinxlineitem{Parameters}
\sphinxAtStartPar
\sphinxstyleliteralstrong{\sphinxupquote{context}} (\sphinxstyleliteralemphasis{\sphinxupquote{\{\}}}\sphinxstyleliteralemphasis{\sphinxupquote{ (}}\sphinxstyleliteralemphasis{\sphinxupquote{set of spacy.Spans = Entities of Entity Ruler}}\sphinxstyleliteralemphasis{\sphinxupquote{)}}) \textendash{} 

\sphinxlineitem{Return type}
\sphinxAtStartPar
None

\end{description}\end{quote}

\end{fulllineitems}


\end{fulllineitems}



\paragraph{medextractor.knowledge.relations module}
\label{\detokenize{medextractor.knowledge:module-medextractor.knowledge.relations}}\label{\detokenize{medextractor.knowledge:medextractor-knowledge-relations-module}}\index{module@\spxentry{module}!medextractor.knowledge.relations@\spxentry{medextractor.knowledge.relations}}\index{medextractor.knowledge.relations@\spxentry{medextractor.knowledge.relations}!module@\spxentry{module}}\index{RelationType (class in medextractor.knowledge.relations)@\spxentry{RelationType}\spxextra{class in medextractor.knowledge.relations}}

\begin{fulllineitems}
\phantomsection\label{\detokenize{medextractor.knowledge:medextractor.knowledge.relations.RelationType}}
\pysigstartsignatures
\pysiglinewithargsret{\sphinxbfcode{\sphinxupquote{class\DUrole{w}{  }}}\sphinxcode{\sphinxupquote{medextractor.knowledge.relations.}}\sphinxbfcode{\sphinxupquote{RelationType}}}{\emph{\DUrole{n}{value}}}{}
\pysigstopsignatures
\sphinxAtStartPar
Bases: \sphinxcode{\sphinxupquote{Enum}}

\sphinxAtStartPar
Types of relations to be used in the KnowledgeBase.
\index{HAS\_SYMPTOM (medextractor.knowledge.relations.RelationType attribute)@\spxentry{HAS\_SYMPTOM}\spxextra{medextractor.knowledge.relations.RelationType attribute}}

\begin{fulllineitems}
\phantomsection\label{\detokenize{medextractor.knowledge:medextractor.knowledge.relations.RelationType.HAS_SYMPTOM}}
\pysigstartsignatures
\pysigline{\sphinxbfcode{\sphinxupquote{HAS\_SYMPTOM}}\sphinxbfcode{\sphinxupquote{\DUrole{w}{  }\DUrole{p}{=}\DUrole{w}{  }2}}}
\pysigstopsignatures
\end{fulllineitems}

\index{IS\_SYMPTOM\_OF (medextractor.knowledge.relations.RelationType attribute)@\spxentry{IS\_SYMPTOM\_OF}\spxextra{medextractor.knowledge.relations.RelationType attribute}}

\begin{fulllineitems}
\phantomsection\label{\detokenize{medextractor.knowledge:medextractor.knowledge.relations.RelationType.IS_SYMPTOM_OF}}
\pysigstartsignatures
\pysigline{\sphinxbfcode{\sphinxupquote{IS\_SYMPTOM\_OF}}\sphinxbfcode{\sphinxupquote{\DUrole{w}{  }\DUrole{p}{=}\DUrole{w}{  }1}}}
\pysigstopsignatures
\end{fulllineitems}

\index{NO\_RELATION (medextractor.knowledge.relations.RelationType attribute)@\spxentry{NO\_RELATION}\spxextra{medextractor.knowledge.relations.RelationType attribute}}

\begin{fulllineitems}
\phantomsection\label{\detokenize{medextractor.knowledge:medextractor.knowledge.relations.RelationType.NO_RELATION}}
\pysigstartsignatures
\pysigline{\sphinxbfcode{\sphinxupquote{NO\_RELATION}}\sphinxbfcode{\sphinxupquote{\DUrole{w}{  }\DUrole{p}{=}\DUrole{w}{  }3}}}
\pysigstopsignatures
\end{fulllineitems}


\end{fulllineitems}



\paragraph{medextractor.knowledge.semantics module}
\label{\detokenize{medextractor.knowledge:module-medextractor.knowledge.semantics}}\label{\detokenize{medextractor.knowledge:medextractor-knowledge-semantics-module}}\index{module@\spxentry{module}!medextractor.knowledge.semantics@\spxentry{medextractor.knowledge.semantics}}\index{medextractor.knowledge.semantics@\spxentry{medextractor.knowledge.semantics}!module@\spxentry{module}}\index{SemanticRelation (class in medextractor.knowledge.semantics)@\spxentry{SemanticRelation}\spxextra{class in medextractor.knowledge.semantics}}

\begin{fulllineitems}
\phantomsection\label{\detokenize{medextractor.knowledge:medextractor.knowledge.semantics.SemanticRelation}}
\pysigstartsignatures
\pysiglinewithargsret{\sphinxbfcode{\sphinxupquote{class\DUrole{w}{  }}}\sphinxcode{\sphinxupquote{medextractor.knowledge.semantics.}}\sphinxbfcode{\sphinxupquote{SemanticRelation}}}{\emph{\DUrole{n}{entity\_1}\DUrole{p}{:}\DUrole{w}{  }\DUrole{n}{{\hyperref[\detokenize{medextractor.knowledge:medextractor.knowledge.entity.Entity}]{\sphinxcrossref{Entity}}}}}, \emph{\DUrole{n}{entity\_2}\DUrole{p}{:}\DUrole{w}{  }\DUrole{n}{{\hyperref[\detokenize{medextractor.knowledge:medextractor.knowledge.entity.Entity}]{\sphinxcrossref{Entity}}}}}, \emph{\DUrole{n}{relation\_type}\DUrole{p}{:}\DUrole{w}{  }\DUrole{n}{{\hyperref[\detokenize{medextractor.knowledge:medextractor.knowledge.relations.RelationType}]{\sphinxcrossref{RelationType}}}}}, \emph{\DUrole{n}{training\_sample}\DUrole{p}{:}\DUrole{w}{  }\DUrole{n}{Optional\DUrole{p}{{[}}str\DUrole{p}{{]}}}\DUrole{w}{  }\DUrole{o}{=}\DUrole{w}{  }\DUrole{default_value}{None}}}{}
\pysigstopsignatures
\sphinxAtStartPar
Bases: \sphinxcode{\sphinxupquote{object}}

\sphinxAtStartPar
A semantic relation between two entities connected by a value of RelationType.

\sphinxAtStartPar
Additionally, training samples can be saved that resulted in this semantic relation.
\index{add\_training\_sample() (medextractor.knowledge.semantics.SemanticRelation method)@\spxentry{add\_training\_sample()}\spxextra{medextractor.knowledge.semantics.SemanticRelation method}}

\begin{fulllineitems}
\phantomsection\label{\detokenize{medextractor.knowledge:medextractor.knowledge.semantics.SemanticRelation.add_training_sample}}
\pysigstartsignatures
\pysiglinewithargsret{\sphinxbfcode{\sphinxupquote{add\_training\_sample}}}{\emph{\DUrole{n}{training\_sample}\DUrole{p}{:}\DUrole{w}{  }\DUrole{n}{str}}}{}
\pysigstopsignatures
\sphinxAtStartPar
Adds the training\_sample into the list of training\_samples.
\begin{quote}\begin{description}
\sphinxlineitem{Parameters}
\sphinxAtStartPar
\sphinxstyleliteralstrong{\sphinxupquote{training\_sample}} (\sphinxstyleliteralemphasis{\sphinxupquote{string}}) \textendash{} The text sample/sentence to be added

\sphinxlineitem{Return type}
\sphinxAtStartPar
None

\end{description}\end{quote}

\end{fulllineitems}

\index{contains\_training\_sample() (medextractor.knowledge.semantics.SemanticRelation method)@\spxentry{contains\_training\_sample()}\spxextra{medextractor.knowledge.semantics.SemanticRelation method}}

\begin{fulllineitems}
\phantomsection\label{\detokenize{medextractor.knowledge:medextractor.knowledge.semantics.SemanticRelation.contains_training_sample}}
\pysigstartsignatures
\pysiglinewithargsret{\sphinxbfcode{\sphinxupquote{contains\_training\_sample}}}{\emph{\DUrole{n}{training\_sample}\DUrole{p}{:}\DUrole{w}{  }\DUrole{n}{str}}}{{ $\rightarrow$ bool}}
\pysigstopsignatures
\sphinxAtStartPar
Checks whether the training\_sample given is already included in the list of training samples.
\begin{quote}\begin{description}
\sphinxlineitem{Parameters}
\sphinxAtStartPar
\sphinxstyleliteralstrong{\sphinxupquote{training\_sample}} (\sphinxstyleliteralemphasis{\sphinxupquote{string}}) \textendash{} A text sample/sentence

\sphinxlineitem{Return type}
\sphinxAtStartPar
true, if training\_sample is contained in the list, false otherwise

\end{description}\end{quote}

\end{fulllineitems}


\end{fulllineitems}



\paragraph{Module contents}
\label{\detokenize{medextractor.knowledge:module-medextractor.knowledge}}\label{\detokenize{medextractor.knowledge:module-contents}}\index{module@\spxentry{module}!medextractor.knowledge@\spxentry{medextractor.knowledge}}\index{medextractor.knowledge@\spxentry{medextractor.knowledge}!module@\spxentry{module}}
\sphinxstepscope


\subsubsection{medextractor.preprocessor package}
\label{\detokenize{medextractor.preprocessor:medextractor-preprocessor-package}}\label{\detokenize{medextractor.preprocessor::doc}}

\paragraph{Submodules}
\label{\detokenize{medextractor.preprocessor:submodules}}

\paragraph{medextractor.preprocessor.preprocessor module}
\label{\detokenize{medextractor.preprocessor:module-medextractor.preprocessor.preprocessor}}\label{\detokenize{medextractor.preprocessor:medextractor-preprocessor-preprocessor-module}}\index{module@\spxentry{module}!medextractor.preprocessor.preprocessor@\spxentry{medextractor.preprocessor.preprocessor}}\index{medextractor.preprocessor.preprocessor@\spxentry{medextractor.preprocessor.preprocessor}!module@\spxentry{module}}\index{RuleBasedPreprocessor (class in medextractor.preprocessor.preprocessor)@\spxentry{RuleBasedPreprocessor}\spxextra{class in medextractor.preprocessor.preprocessor}}

\begin{fulllineitems}
\phantomsection\label{\detokenize{medextractor.preprocessor:medextractor.preprocessor.preprocessor.RuleBasedPreprocessor}}
\pysigstartsignatures
\pysiglinewithargsret{\sphinxbfcode{\sphinxupquote{class\DUrole{w}{  }}}\sphinxcode{\sphinxupquote{medextractor.preprocessor.preprocessor.}}\sphinxbfcode{\sphinxupquote{RuleBasedPreprocessor}}}{\emph{\DUrole{n}{doc\_name}}, \emph{\DUrole{n}{with\_pysbd}}}{}
\pysigstopsignatures
\sphinxAtStartPar
Bases: \sphinxcode{\sphinxupquote{object}}
\index{get\_preprocessed\_text() (medextractor.preprocessor.preprocessor.RuleBasedPreprocessor method)@\spxentry{get\_preprocessed\_text()}\spxextra{medextractor.preprocessor.preprocessor.RuleBasedPreprocessor method}}

\begin{fulllineitems}
\phantomsection\label{\detokenize{medextractor.preprocessor:medextractor.preprocessor.preprocessor.RuleBasedPreprocessor.get_preprocessed_text}}
\pysigstartsignatures
\pysiglinewithargsret{\sphinxbfcode{\sphinxupquote{get\_preprocessed\_text}}}{}{{ $\rightarrow$ str}}
\pysigstopsignatures
\sphinxAtStartPar
Reads the text given in the document with which the Preprocessor is initialised and
processes this text such that it is in a good format for further processing.
\begin{quote}\begin{description}
\sphinxlineitem{Parameters}
\sphinxAtStartPar
\sphinxstyleliteralstrong{\sphinxupquote{None}} \textendash{} 

\sphinxlineitem{Return type}
\sphinxAtStartPar
string

\end{description}\end{quote}

\end{fulllineitems}

\index{pysbd\_sentence\_boundaries() (medextractor.preprocessor.preprocessor.RuleBasedPreprocessor method)@\spxentry{pysbd\_sentence\_boundaries()}\spxextra{medextractor.preprocessor.preprocessor.RuleBasedPreprocessor method}}

\begin{fulllineitems}
\phantomsection\label{\detokenize{medextractor.preprocessor:medextractor.preprocessor.preprocessor.RuleBasedPreprocessor.pysbd_sentence_boundaries}}
\pysigstartsignatures
\pysiglinewithargsret{\sphinxbfcode{\sphinxupquote{pysbd\_sentence\_boundaries}}}{}{}
\pysigstopsignatures
\sphinxAtStartPar
Creates a SpaCy pipeline component to segment a text into sentences using pysbd.
\begin{quote}\begin{description}
\sphinxlineitem{Parameters}
\sphinxAtStartPar
\sphinxstyleliteralstrong{\sphinxupquote{doc}} (\sphinxstyleliteralemphasis{\sphinxupquote{Doc}}) \textendash{} SpaCy Doc object

\sphinxlineitem{Returns}
\sphinxAtStartPar
\sphinxstylestrong{doc} \textendash{} SpaCy Doc object

\sphinxlineitem{Return type}
\sphinxAtStartPar
Doc

\end{description}\end{quote}

\end{fulllineitems}


\end{fulllineitems}



\paragraph{Module contents}
\label{\detokenize{medextractor.preprocessor:module-medextractor.preprocessor}}\label{\detokenize{medextractor.preprocessor:module-contents}}\index{module@\spxentry{module}!medextractor.preprocessor@\spxentry{medextractor.preprocessor}}\index{medextractor.preprocessor@\spxentry{medextractor.preprocessor}!module@\spxentry{module}}
\sphinxstepscope


\subsubsection{medextractor.rdf package}
\label{\detokenize{medextractor.rdf:medextractor-rdf-package}}\label{\detokenize{medextractor.rdf::doc}}

\paragraph{Submodules}
\label{\detokenize{medextractor.rdf:submodules}}

\paragraph{medextractor.rdf.RDFSerialiser module}
\label{\detokenize{medextractor.rdf:module-medextractor.rdf.RDFSerialiser}}\label{\detokenize{medextractor.rdf:medextractor-rdf-rdfserialiser-module}}\index{module@\spxentry{module}!medextractor.rdf.RDFSerialiser@\spxentry{medextractor.rdf.RDFSerialiser}}\index{medextractor.rdf.RDFSerialiser@\spxentry{medextractor.rdf.RDFSerialiser}!module@\spxentry{module}}\index{RDFSerialiser (class in medextractor.rdf.RDFSerialiser)@\spxentry{RDFSerialiser}\spxextra{class in medextractor.rdf.RDFSerialiser}}

\begin{fulllineitems}
\phantomsection\label{\detokenize{medextractor.rdf:medextractor.rdf.RDFSerialiser.RDFSerialiser}}
\pysigstartsignatures
\pysiglinewithargsret{\sphinxbfcode{\sphinxupquote{class\DUrole{w}{  }}}\sphinxcode{\sphinxupquote{medextractor.rdf.RDFSerialiser.}}\sphinxbfcode{\sphinxupquote{RDFSerialiser}}}{\emph{\DUrole{n}{knowledgebase}}, \emph{\DUrole{n}{namespace}}, \emph{\DUrole{n}{namespace\_prefix}}}{}
\pysigstopsignatures
\sphinxAtStartPar
Bases: \sphinxcode{\sphinxupquote{object}}
\index{knowledgebase\_to\_graph() (medextractor.rdf.RDFSerialiser.RDFSerialiser method)@\spxentry{knowledgebase\_to\_graph()}\spxextra{medextractor.rdf.RDFSerialiser.RDFSerialiser method}}

\begin{fulllineitems}
\phantomsection\label{\detokenize{medextractor.rdf:medextractor.rdf.RDFSerialiser.RDFSerialiser.knowledgebase_to_graph}}
\pysigstartsignatures
\pysiglinewithargsret{\sphinxbfcode{\sphinxupquote{knowledgebase\_to\_graph}}}{}{}
\pysigstopsignatures
\sphinxAtStartPar
Transfers the content of the knowledgebase into an rdflib\sphinxhyphen{}graph.
\begin{quote}\begin{description}
\sphinxlineitem{Parameters}
\sphinxAtStartPar
\sphinxstyleliteralstrong{\sphinxupquote{None}} \textendash{} 

\sphinxlineitem{Return type}
\sphinxAtStartPar
None

\end{description}\end{quote}

\end{fulllineitems}

\index{serialise\_knowledgebase() (medextractor.rdf.RDFSerialiser.RDFSerialiser method)@\spxentry{serialise\_knowledgebase()}\spxextra{medextractor.rdf.RDFSerialiser.RDFSerialiser method}}

\begin{fulllineitems}
\phantomsection\label{\detokenize{medextractor.rdf:medextractor.rdf.RDFSerialiser.RDFSerialiser.serialise_knowledgebase}}
\pysigstartsignatures
\pysiglinewithargsret{\sphinxbfcode{\sphinxupquote{serialise\_knowledgebase}}}{\emph{\DUrole{n}{output\_path}}}{}
\pysigstopsignatures
\sphinxAtStartPar
Serialises knowledge base into an RDF file.
\begin{quote}\begin{description}
\sphinxlineitem{Parameters}
\sphinxAtStartPar
\sphinxstyleliteralstrong{\sphinxupquote{output\_path}} (\sphinxstyleliteralemphasis{\sphinxupquote{string}}) \textendash{} path to which the xml\sphinxhyphen{}document resulting from the rdflib\sphinxhyphen{}Graph will be saved

\sphinxlineitem{Return type}
\sphinxAtStartPar
None

\end{description}\end{quote}

\end{fulllineitems}


\end{fulllineitems}



\paragraph{medextractor.rdf.graphmanager module}
\label{\detokenize{medextractor.rdf:module-medextractor.rdf.graphmanager}}\label{\detokenize{medextractor.rdf:medextractor-rdf-graphmanager-module}}\index{module@\spxentry{module}!medextractor.rdf.graphmanager@\spxentry{medextractor.rdf.graphmanager}}\index{medextractor.rdf.graphmanager@\spxentry{medextractor.rdf.graphmanager}!module@\spxentry{module}}\index{GraphManager (class in medextractor.rdf.graphmanager)@\spxentry{GraphManager}\spxextra{class in medextractor.rdf.graphmanager}}

\begin{fulllineitems}
\phantomsection\label{\detokenize{medextractor.rdf:medextractor.rdf.graphmanager.GraphManager}}
\pysigstartsignatures
\pysiglinewithargsret{\sphinxbfcode{\sphinxupquote{class\DUrole{w}{  }}}\sphinxcode{\sphinxupquote{medextractor.rdf.graphmanager.}}\sphinxbfcode{\sphinxupquote{GraphManager}}}{\emph{\DUrole{n}{namespace\_prefix}}, \emph{\DUrole{n}{namespace\_uri}}}{}
\pysigstopsignatures
\sphinxAtStartPar
Bases: \sphinxcode{\sphinxupquote{object}}
\index{add\_symptom() (medextractor.rdf.graphmanager.GraphManager method)@\spxentry{add\_symptom()}\spxextra{medextractor.rdf.graphmanager.GraphManager method}}

\begin{fulllineitems}
\phantomsection\label{\detokenize{medextractor.rdf:medextractor.rdf.graphmanager.GraphManager.add_symptom}}
\pysigstartsignatures
\pysiglinewithargsret{\sphinxbfcode{\sphinxupquote{add\_symptom}}}{\emph{\DUrole{n}{disease}}, \emph{\DUrole{n}{symptom}}}{}
\pysigstopsignatures
\sphinxAtStartPar
Adds the given symptom together with the given disease to the rdflib\sphinxhyphen{}graph
and saves the given disease in the set of diseases.
\begin{quote}\begin{description}
\sphinxlineitem{Parameters}\begin{itemize}
\item {} 
\sphinxAtStartPar
\sphinxstyleliteralstrong{\sphinxupquote{disease}} (\sphinxstyleliteralemphasis{\sphinxupquote{string}}) \textendash{} 

\item {} 
\sphinxAtStartPar
\sphinxstyleliteralstrong{\sphinxupquote{symptom}} (\sphinxstyleliteralemphasis{\sphinxupquote{string}}) \textendash{} 

\end{itemize}

\sphinxlineitem{Return type}
\sphinxAtStartPar
None

\end{description}\end{quote}

\end{fulllineitems}

\index{get\_serialized\_graph() (medextractor.rdf.graphmanager.GraphManager method)@\spxentry{get\_serialized\_graph()}\spxextra{medextractor.rdf.graphmanager.GraphManager method}}

\begin{fulllineitems}
\phantomsection\label{\detokenize{medextractor.rdf:medextractor.rdf.graphmanager.GraphManager.get_serialized_graph}}
\pysigstartsignatures
\pysiglinewithargsret{\sphinxbfcode{\sphinxupquote{get\_serialized\_graph}}}{\emph{\DUrole{n}{output\_path}}, \emph{\DUrole{n}{serialization\_format}\DUrole{o}{=}\DUrole{default_value}{\textquotesingle{}pretty\sphinxhyphen{}xml\textquotesingle{}}}}{}
\pysigstopsignatures
\sphinxAtStartPar
Serializes the graph according to the given serialization\_format and
saves the resulting document to the given output\_path.
\begin{quote}\begin{description}
\sphinxlineitem{Parameters}\begin{itemize}
\item {} 
\sphinxAtStartPar
\sphinxstyleliteralstrong{\sphinxupquote{output\_path}} (\sphinxstyleliteralemphasis{\sphinxupquote{string}}) \textendash{} 

\item {} 
\sphinxAtStartPar
\sphinxstyleliteralstrong{\sphinxupquote{serialization\_format}} (\sphinxstyleliteralemphasis{\sphinxupquote{string}}) \textendash{} 

\end{itemize}

\sphinxlineitem{Return type}
\sphinxAtStartPar
None

\end{description}\end{quote}

\end{fulllineitems}


\end{fulllineitems}



\paragraph{Module contents}
\label{\detokenize{medextractor.rdf:module-medextractor.rdf}}\label{\detokenize{medextractor.rdf:module-contents}}\index{module@\spxentry{module}!medextractor.rdf@\spxentry{medextractor.rdf}}\index{medextractor.rdf@\spxentry{medextractor.rdf}!module@\spxentry{module}}

\subsection{Submodules}
\label{\detokenize{medextractor:submodules}}

\subsection{medextractor.config\_manager module}
\label{\detokenize{medextractor:module-medextractor.config_manager}}\label{\detokenize{medextractor:medextractor-config-manager-module}}\index{module@\spxentry{module}!medextractor.config\_manager@\spxentry{medextractor.config\_manager}}\index{medextractor.config\_manager@\spxentry{medextractor.config\_manager}!module@\spxentry{module}}\index{ConfigManager (class in medextractor.config\_manager)@\spxentry{ConfigManager}\spxextra{class in medextractor.config\_manager}}

\begin{fulllineitems}
\phantomsection\label{\detokenize{medextractor:medextractor.config_manager.ConfigManager}}
\pysigstartsignatures
\pysigline{\sphinxbfcode{\sphinxupquote{class\DUrole{w}{  }}}\sphinxcode{\sphinxupquote{medextractor.config\_manager.}}\sphinxbfcode{\sphinxupquote{ConfigManager}}}
\pysigstopsignatures
\sphinxAtStartPar
Bases: \sphinxcode{\sphinxupquote{object}}

\end{fulllineitems}



\subsection{medextractor.create\_manual\_rdf\_graph module}
\label{\detokenize{medextractor:module-medextractor.create_manual_rdf_graph}}\label{\detokenize{medextractor:medextractor-create-manual-rdf-graph-module}}\index{module@\spxentry{module}!medextractor.create\_manual\_rdf\_graph@\spxentry{medextractor.create\_manual\_rdf\_graph}}\index{medextractor.create\_manual\_rdf\_graph@\spxentry{medextractor.create\_manual\_rdf\_graph}!module@\spxentry{module}}
\sphinxAtStartPar
Program for “manual” creation of RDF\sphinxhyphen{}Graphs:
Disease and corresponding symptoms are given, an RDF\sphinxhyphen{}Graph is built and saved as xml\sphinxhyphen{}file.
\index{add\_symptom() (in module medextractor.create\_manual\_rdf\_graph)@\spxentry{add\_symptom()}\spxextra{in module medextractor.create\_manual\_rdf\_graph}}

\begin{fulllineitems}
\phantomsection\label{\detokenize{medextractor:medextractor.create_manual_rdf_graph.add_symptom}}
\pysigstartsignatures
\pysiglinewithargsret{\sphinxcode{\sphinxupquote{medextractor.create\_manual\_rdf\_graph.}}\sphinxbfcode{\sphinxupquote{add\_symptom}}}{\emph{\DUrole{n}{symptom\_str}\DUrole{p}{:}\DUrole{w}{  }\DUrole{n}{str}}}{}
\pysigstopsignatures
\end{fulllineitems}



\subsection{medextractor.medextractor module}
\label{\detokenize{medextractor:module-medextractor.medextractor}}\label{\detokenize{medextractor:medextractor-medextractor-module}}\index{module@\spxentry{module}!medextractor.medextractor@\spxentry{medextractor.medextractor}}\index{medextractor.medextractor@\spxentry{medextractor.medextractor}!module@\spxentry{module}}

\subsection{medextractor.ruler\_creator module}
\label{\detokenize{medextractor:module-medextractor.ruler_creator}}\label{\detokenize{medextractor:medextractor-ruler-creator-module}}\index{module@\spxentry{module}!medextractor.ruler\_creator@\spxentry{medextractor.ruler\_creator}}\index{medextractor.ruler\_creator@\spxentry{medextractor.ruler\_creator}!module@\spxentry{module}}\index{RulerCreator (class in medextractor.ruler\_creator)@\spxentry{RulerCreator}\spxextra{class in medextractor.ruler\_creator}}

\begin{fulllineitems}
\phantomsection\label{\detokenize{medextractor:medextractor.ruler_creator.RulerCreator}}
\pysigstartsignatures
\pysigline{\sphinxbfcode{\sphinxupquote{class\DUrole{w}{  }}}\sphinxcode{\sphinxupquote{medextractor.ruler\_creator.}}\sphinxbfcode{\sphinxupquote{RulerCreator}}}
\pysigstopsignatures
\sphinxAtStartPar
Bases: \sphinxcode{\sphinxupquote{object}}
\index{load() (medextractor.ruler\_creator.RulerCreator method)@\spxentry{load()}\spxextra{medextractor.ruler\_creator.RulerCreator method}}

\begin{fulllineitems}
\phantomsection\label{\detokenize{medextractor:medextractor.ruler_creator.RulerCreator.load}}
\pysigstartsignatures
\pysiglinewithargsret{\sphinxbfcode{\sphinxupquote{load}}}{}{}
\pysigstopsignatures
\end{fulllineitems}

\index{save() (medextractor.ruler\_creator.RulerCreator method)@\spxentry{save()}\spxextra{medextractor.ruler\_creator.RulerCreator method}}

\begin{fulllineitems}
\phantomsection\label{\detokenize{medextractor:medextractor.ruler_creator.RulerCreator.save}}
\pysigstartsignatures
\pysiglinewithargsret{\sphinxbfcode{\sphinxupquote{save}}}{}{{ $\rightarrow$ None}}
\pysigstopsignatures
\end{fulllineitems}


\end{fulllineitems}



\subsection{Module contents}
\label{\detokenize{medextractor:module-medextractor}}\label{\detokenize{medextractor:module-contents}}\index{module@\spxentry{module}!medextractor@\spxentry{medextractor}}\index{medextractor@\spxentry{medextractor}!module@\spxentry{module}}
\sphinxAtStartPar
Main Medextractor

\sphinxAtStartPar
This is the form of a docstring.

\sphinxAtStartPar
It can be spread over several lines.


\chapter{Indices and tables}
\label{\detokenize{index:indices-and-tables}}\begin{itemize}
\item {} 
\sphinxAtStartPar
\DUrole{xref,std,std-ref}{genindex}

\item {} 
\sphinxAtStartPar
\DUrole{xref,std,std-ref}{modindex}

\item {} 
\sphinxAtStartPar
\DUrole{xref,std,std-ref}{search}

\end{itemize}


\renewcommand{\indexname}{Python Module Index}
\begin{sphinxtheindex}
\let\bigletter\sphinxstyleindexlettergroup
\bigletter{m}
\item\relax\sphinxstyleindexentry{medextractor}\sphinxstyleindexpageref{medextractor:\detokenize{module-medextractor}}
\item\relax\sphinxstyleindexentry{medextractor.config\_manager}\sphinxstyleindexpageref{medextractor:\detokenize{module-medextractor.config_manager}}
\item\relax\sphinxstyleindexentry{medextractor.create\_manual\_rdf\_graph}\sphinxstyleindexpageref{medextractor:\detokenize{module-medextractor.create_manual_rdf_graph}}
\item\relax\sphinxstyleindexentry{medextractor.knowledge}\sphinxstyleindexpageref{medextractor.knowledge:\detokenize{module-medextractor.knowledge}}
\item\relax\sphinxstyleindexentry{medextractor.knowledge.base}\sphinxstyleindexpageref{medextractor.knowledge:\detokenize{module-medextractor.knowledge.base}}
\item\relax\sphinxstyleindexentry{medextractor.knowledge.entity}\sphinxstyleindexpageref{medextractor.knowledge:\detokenize{module-medextractor.knowledge.entity}}
\item\relax\sphinxstyleindexentry{medextractor.knowledge.knowledge\_extractor}\sphinxstyleindexpageref{medextractor.knowledge:\detokenize{module-medextractor.knowledge.knowledge_extractor}}
\item\relax\sphinxstyleindexentry{medextractor.knowledge.relations}\sphinxstyleindexpageref{medextractor.knowledge:\detokenize{module-medextractor.knowledge.relations}}
\item\relax\sphinxstyleindexentry{medextractor.knowledge.semantics}\sphinxstyleindexpageref{medextractor.knowledge:\detokenize{module-medextractor.knowledge.semantics}}
\item\relax\sphinxstyleindexentry{medextractor.medextractor}\sphinxstyleindexpageref{medextractor:\detokenize{module-medextractor.medextractor}}
\item\relax\sphinxstyleindexentry{medextractor.preprocessor}\sphinxstyleindexpageref{medextractor.preprocessor:\detokenize{module-medextractor.preprocessor}}
\item\relax\sphinxstyleindexentry{medextractor.preprocessor.preprocessor}\sphinxstyleindexpageref{medextractor.preprocessor:\detokenize{module-medextractor.preprocessor.preprocessor}}
\item\relax\sphinxstyleindexentry{medextractor.rdf}\sphinxstyleindexpageref{medextractor.rdf:\detokenize{module-medextractor.rdf}}
\item\relax\sphinxstyleindexentry{medextractor.rdf.graphmanager}\sphinxstyleindexpageref{medextractor.rdf:\detokenize{module-medextractor.rdf.graphmanager}}
\item\relax\sphinxstyleindexentry{medextractor.rdf.RDFSerialiser}\sphinxstyleindexpageref{medextractor.rdf:\detokenize{module-medextractor.rdf.RDFSerialiser}}
\item\relax\sphinxstyleindexentry{medextractor.ruler\_creator}\sphinxstyleindexpageref{medextractor:\detokenize{module-medextractor.ruler_creator}}
\end{sphinxtheindex}

\renewcommand{\indexname}{Index}
\printindex
\end{document}