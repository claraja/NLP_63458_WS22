\chapter{Zusammenfassung und Ausblick}
\label{ch:zusammenfassung}



In diesem Kapitel wird die Praktikumsarbeit zusammengefasst und ein Ausblick auf sinnvolle weiterführende Arbeiten gegeben.




\section{Zusammenfassung}
\label{sec:Zusammenfassung} 

In der vorliegenden Arbeit wurde die prototypische \emph{MedExtractor}-Software entwickelt, mit der automatisiert Wissensrepräsentationen englischsprachiger medizinischer Texte erstellt werden können. Diese Wissensrepräsentationen sollen dabei als Vorarbeit für einen Chatbot dienen, der Menschen mit psychischen Problemen beraten können soll. Im Rahmen dieses Praktikumsbeitrages beschränken sich die Wissensrepräsentationen auf die Erstellung einer Liste von psychologischen Krankheiten und ihren Symptomen. 

Die prototypische Software wurde in Python geschrieben und verwendet \emph{spaCy} als  auf \emph{Natural Language Processing (NLP)} spezialisierte Programmbibliothek. Der Schwerpunkt lag dabei auf der Einbindung einer \emph{Named Entity Recognition (NER)}-Komponente zur Erkennung und Kategorisierung von medizinischen Fachbegriffen.

Als Quelle für medizinische Fachbegriffe diente das \emph{MetaMapLite}-Projekt der \emph{National Library of Medicine}, das Tausende von Einträgen enthält und insbesondere zahlreiche Begriffe als \emph{disorder/disease} bzw. \emph{finding} kategorisiert und damit für das Training einer \emph{NER}-Komponente verwendet werden kann. Als \emph{NER}-Komponente von \emph{spaCy} wurde der \emph{Entity Ruler} ausgewählt, da dieser die trainierten Begriffe konventionell in einem Text sucht und kein statistisches mit Beispielsätzen (welche zu Beginn des Praktikums nicht vorhanden waren) trainiertes Modell verwendet.

Die prototypische Software enthält drei zentrale Module. Zum einen ist dies ein \emph{Preprocessor}, der den zu analysierenden Text vorstrukturiert, so dass dieser einfacher analysiert werden kann - z.B. durch Umwandlung von Aufzählungen in eine Reihe von Sätzen, die für sich allein stehen können. In dem so vorbereiteten Text sucht anschließend das Hauptmodul der Software, der mit dem medizinischem Fachvokabular vortrainierte \emph{KnowledgeExtractor}, nach Symptomen und Krankheiten und bringt diese in einen Zusammenhang, wenn sie in demselben Satz gefunden werden, und speichert diese in einer programminternen Wissensbasis (\emph{KnowledgeBase}) ab. Abschließend serialisiert das \emph{RDFSerialiser}-Modul die Wissensrepräsentation in Form eines \emph{Resource Description Frameworks (RDF)}.

Der \emph{MedExtractor} wurde mit zahlreichen Texten getestet und erzielte in mehreren Texten trotz seiner sehr einfachen Logik eine sehr hohe Präzision, d.h. die vom \emph{MedExtractor} gefundenen Zusammenhänge waren auch in manuell erstellten Wissensrepräsentationen der Texte enthalten. Etwas weniger erfolgreich war der \emph{MedExtractor} hinsichtlich der Sensitivität, also der Vollständigkeit der gefundenen Zusammenhänge.

Ferner wurden in der Evaluierung die Gründe analysiert, aus denen der MedExtractor keine optimalen Werte bei Präzision und Sensitivität erzielte. Zu den Gründen zählen unspezifische oder mehrdeutige Begriffe im Trainingsvokabular, die Inflexibilität des \emph{Entity Rulers} sowie die mangelnde Fähigkeit des MedExtractors, die Aussagen von Sätzen richtig zu erfassen.

\section{Ausblick}
\label{sec:Ausblick} 

Da das Trainingsvokabular bereits sehr umfangreich ist, wird eine Verbesserung des MedExtractors hauptsächlich dadurch zu erzielen sein, dass in zukünftigen Projekten die grammatikalische Analyse durch die \emph{spaCy}-Pipeline genutzt wird. Dadurch können einfache Verben und Adjektive im Trainingsvokabular erkannt und ggf. verworfen werden. Insbesondere erlaubt es die Satzanalyse, typische Satzkonstrukte, mit denen Symptome von Krankheiten ausgedrückt werden, gezielt zu suchen und damit der reinen \emph{Named Entity Recognition} eine qualitativ anders geartete Methode zur Seite zu stellen.

Wie gezeigt, bedarf insbesondere die Sensitivität des MedExtractors einer Verbesserung, da viele Zusammenhänge nicht gefunden werden, da verwendete Begriffe nicht oder nur in leicht abgewandelter Form im Trainingsvokabular enthalten sind. Die Suche nach typischen Satzkonstrukten dürfte insbesondere für die Verbesserung der Sensitivität von großer Bedeutung sein, da dann auch Krankheiten und Symptome gefunden werden, die im Trainingsvokabular nicht enthalten sind.

Eine der Stärken des MedExtractors ist die Sammlung von Trainingssätzen. Die im Rahmen dieses Praktikums gesammelte Menge an Trainingssätzen reicht jedoch nicht aus, um zuverlässig statistische Modelle trainieren zu können. Hier sollten weitere Texte analysiert werden, um den Pool an Trainingssätzen zu erweitern mit dem Ziel, die \emph{Named Entity Recognition} kontextsensitiver zu gestalten. 