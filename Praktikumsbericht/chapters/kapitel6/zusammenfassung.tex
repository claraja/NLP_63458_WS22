\chapter{Zusammenfassung und Ausblick}
\label{ch:zusammenfassung}


------

- Zusammenfassung der Arbeit und der Ergebnisse (kritisch betrachtet) und Ausblick auf weiterführende Fragestellungen zum Thema.


- Eine Bewertung der Ergebnisse und der Bedeutung der Ergebnisse im wissenschaftlichen Umfeld.


- Offen Fragen sollten benannt werden und bieten die Gelegenheit, einen Ausblick auf sinnvolle weiterführende Arbeiten zu geben.

-----


In dieser Praktikumsarbeit wurde eine prototypische Konsolenapplikation entwickelt, die englischsprachige medizinische Texte analysiert und das darin enthaltene Wissen strukturiert in einem RDF-Graphen hinterlegt. Die in RDF-Form gespeicherte Wissensbasis soll nachfolgenden Programmen, wie beispielsweise medizinischen Chatbots, als Wissensgrundlage für Argumentationen und Empfehlungen dienen können.

Die Applikation \emph{MedExtractor} besteht aus drei Hauptkomponenten. Einer Komponente zur Vorverarbeitung von Texten, die die zu verarbeitenden Texte so vorbereitet, dass die Inhalte für die nachfolgende Komponente gut zu verarbeiten sind. Die Hauptkomponente wird mit Begriffen aus einer medizinischen Datenbank trainiert, extrahiert dann aus den zu verarbeitenden Texten Krankheiten und Symptome und speichert diese in einer programminternen Wissensbasis ab. Die letzte Komponente in der Applikationspipeline serialisiert die Wissensbasis in Form einer RDF-Datei.


\section{Zusammenfassung}
\label{sec:Zusammenfassung} 

\section{Ausblick}
\label{sec:Ausblick} 