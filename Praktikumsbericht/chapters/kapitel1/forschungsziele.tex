
\section{Forschungsziele}

Dieser Abschnitt beschreibt die Forschungsziele, die sich aus den in Abschnitt \ref{sec:forschungsfragen} angeführten Problemen ableiten.


\begin{enumerate}[label=FZ \arabic*]
\item Wie kann ein medizinischer Text vorverarbeitet werden, so dass relevante Aussagen (Überschriften, Aufzählungen, Leerzeilen und Dialoge) automatisch erkannt werden?

\begin{enumerate}[label=\theenumi.\arabic*]
\item Aufbau und Struktur medizinischer Texte (Observation)
\item Theoriebildung zur Vorverarbeitung medizinischer Texte
\item Implementierung zur Vorverarbeitung medizinischer Texte 
\item Evaluation zur Vorverarbeitung medizinischer Texte
\end{enumerate}

\item Wie kann ein medizinisches Fachvokabular über depressive Erkrankungen automatisiert in eine maschinenlesbare Form überführt werden?

\begin{enumerate}[label=\theenumi.\arabic*]
\item Überblick über depressive Erkrankungen (Observation)
\item Automatisierung durch NLP (Observation)
\item Theoriebildung zur Überführung medizinischen Fachvokabulars in maschinenlesbare Form zur weiteren Verarbeitung durch NLP 
\item Implementierung zur Überführung medizinischen Fachvokabulars in maschinenlesbare Form zur weiteren Verarbeitung durch NLP
\item Evaluation zur Überführung medizinischen Fachvokabulars in maschinenlesbare Form zur weiteren Verarbeitung durch NLP

\end{enumerate}

\item Wie kann eine Wissensrepräsentation über einen gegebenen Text maschinenlesbar erstellt werden?

\begin{enumerate}[label=\theenumi.\arabic*]
\item Wissensrepräsentation mittels RDF (Observation)
\item Theoriebildung zur Wissensrepräsentation
\item Implementierung zur Wissensrepräsentation 
\item Evaluation zur Wissensrepräsentation
\end{enumerate}

\end{enumerate}

