\automark[section]{chapter}
\renewcommand{\chaptermark}[1]{\markboth{\spacedlowsmallcaps{#1}}{\spacedlowsmallcaps{#1}}}
\renewcommand{\sectionmark}[1]{\markright{\thesection\enspace\spacedlowsmallcaps{#1}}}
\refstepcounter{dummy}

\chapter{Einleitung}

Diese Praktikumsarbeit behandelt die automatische Erstellung einer Wissensrepräsentation aus einem medizinischen Text. Eine prototypische Softwarelösung wird entwickelt, die für einen vorgegebenen Beispieltext einen Wissensgraphen ermittelt.

Gemäß einem Artikel des \emph{Economist} von 2017 \cite{the_economist_worlds_2017}, der in Verbindung mit den aktuellen technischen Entwicklungen, Künstlicher Intelligenz und Data Science viel zitiert wird, ist nicht mehr Öl die wertvollste Ressource, sondern Daten. Ebenso wie Öl müssen die Daten jedoch erst aufbereitet werden, um tatsächlich von Nutzen zu sein. Auch Informationen, die in Form von für den Menschen lesbaren Texten zur Verfügung stehen, sind nicht direkt für Computer auswertbar. Daher wurden mit den Methoden des \emph{Natural Language Processing} (NLP) Verfahren entwickelt, um automatisch Informationen aus Texten extrahieren zu können. Die allgemeinsprachlichen NLP-Verfahren bedürfen jedoch noch einer Spezialisierung, um auch für Fachtexte wie zum Beispiel aus dem Bereich der Medizin nutzbringend angewendet werden zu können.

Aus diesen Rahmenbedingungen ergibt sich die Motivation und die Problembeschreibung für diese Praktikumsaufgabe, die in den beiden nachfolgenden Abschnitten beschrieben werden.