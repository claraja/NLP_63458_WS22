\chapter{Konzeptuelle Modellierung und Entwurf}
\label{ch:modellierung}
In diesem Kapitel wird der Anwendungskontext und konkrete Anwendungsfälle nach UCSD sowie Modelle der technischen Umsetzung einer Automatisierungsunterstützung (Informationsmodell, Komponenten-/Dienstemodelle, Architekturmodell) entsprechend RUP modelliert.

\section{User Centered System Design }

Der Anwendungskontext und die Anwendungsfälle werden mithilfe des Ansatzes des User Centered System Design nach (UCSD, (Norman & Draper 1986)) ermittelt. Auf der Grundlage der Anwendungsfälle wird das Informations- und Datenmodell entwickelt und daraus das Komponenten- und Architekturmodell.
Gemäß Aufgabenstellung umfasst die Entwicklung keine graphische Benutzeroberfläche.

\subsection{Anwendungskontext und Anwendungsfall}

Der Anwendungskontext ist eine Unterstützung bei der Behandlung und Betreuung psychisch erkrankter Personen durch Chatbots. Um den Chatbots den erforderlichen Kontext und das entsprechende Hintergrundwissen zu vermitteln, wird eine Wissensrepräsentation über die psychischen Erkrankungen benötigt. Die zu entwickelnde Konsolenapplikation ist dazu gedacht, Forschenden und später auch Chatbots zu ermöglichen, aus einem Input in Form einer reinen Textdatei die Wissensrepräsentation zu erstellen. Dazu soll die Applikation mit dem Namen des Textes als Parameter aufgerufen werden. Das Ergebnis der Applikationsausführung ist eine RDF-Datei.

\subsection{Informationsmodell}

\subsection{Datenmodell}

\subsection{Architekturmodell}

Kommandozeileninterface
        |
Geschäftslogik
        |
Gespeicherte Pipeline


\section{FZ 1.2 Theoriebildung zur Vorverarbeitung medizinischer Texte}
\label{sec:FZ1.2} 

\section{FZ 2.3 Theoriebildung zur Überführung medizinischen Fachvokabulars in maschinenlesbare Form zur weiteren Verarbeitung durch NLP}
\label{sec:FZ2.3} 

\section{FZ 3.2 Theoriebildung zur Wissensrepräsentation}
\label{sec:FZ3.2} 
