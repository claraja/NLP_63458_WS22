\chapter{Konzeptuelle Modellierung und Entwurf}
\label{ch:modellierung}
In diesem Kapitel wird der Anwendungskontext und konkrete Anwendungsfälle nach UCSD sowie Modelle der technischen Umsetzung einer Automatisierungsunterstützung (Informationsmodell, Komponenten-/Dienstemodelle, Architekturmodell) entsprechend RUP modelliert.

\section{User Centered System Design }

Der Anwendungskontext und die Anwendungsfälle werden mithilfe des Ansatzes des User Centered System Design nach \cite{norman1986user} ermittelt. Auf der Grundlage der Anwendungsfälle wird das Informations- und Datenmodell entwickelt und daraus das Komponenten- und Architekturmodell.


Gemäß Aufgabenstellung umfasst die Entwicklung keine graphische Benutzeroberfläche.

\subsection{Anwendungskontext und Anwendungsfall}

Der Anwendungskontext ist eine Unterstützung bei der Behandlung und Betreuung psychisch erkrankter Personen durch Chatbots. Um den Chatbots den erforderlichen Kontext und das entsprechende Hintergrundwissen zu vermitteln, wird eine Wissensrepräsentation über die psychischen Erkrankungen benötigt. Die zu entwickelnde Konsolenapplikation \emph{MedExtractor} ist dazu gedacht, Forschenden und später auch Chatbots zu ermöglichen, aus einem Input in Form einer reinen Textdatei die Wissensrepräsentation zu erstellen, siehe Abbildung \ref{fig:anwendungsfaelle}. Dazu soll die Applikation mit dem Namen des Textes als Parameter aufgerufen werden. Das Ergebnis der Applikationsausführung ist eine RDF-Datei. 

Als zusätzliche Funktionalität ist denkbar, die Wissensbasis in Form der trainierten spaCy-EntityLinker-Repräsentation auszugeben.


\begin{figure}
    \centering
    \includegraphics[width=\textwidth]{pictures/UseCases.png}
    \caption{Anwendungsfall der Konsolenapplikation}
    \label{fig:anwendungsfaelle}
\end{figure}

\subsection{Aktivitätsmodell}

Für die Modellierung des Anwendungsfalls sind die in Abbildung \ref{fig:Aktivitätsmodell} dargestellten Aktivitäten erforderlich.

\begin{figure}[]
    \centering
    \includegraphics[width=\textwidth]{pictures/Aktivitätsdiagramm.png}
    \caption{Aktivitätsdiagramm der Konsolenapplikation}
    \label{fig:Aktivitätsmodell}
\end{figure}

\subsection{Informationsmodell}

Für den Aufbau der Wissensbasis sind erstens die gefundenen Entitäten und zweitens die Beziehungen zwischen diesen von Bedeutung. Diese werden in Form von semantischen Beziehungen in der Wissensbasis gespeichert, wie dies in  Abbildung \ref{fig:informationsmodell} des Informationsmodells (ER-Diagramm) der Applikation dargestellt ist.

\begin{figure}[h]
    \centering
    \includegraphics[width=\textwidth]{pictures/Informationsmodell.png}
    \caption{Informationsmodell der Konsolenapplikation}
    \label{fig:informationsmodell}
\end{figure}

\subsection{Architektur- und Komponentenmodell}

Aus dem Anwendungsfall und dem Aktivitätsmodell ergibt sich die grobe Gesamtarchitektur der Applikation wie in Abbildung \ref{fig:architekturmodell} dargestellt.

\begin{figure}[h]
    \centering
    \includegraphics[width=\textwidth]{pictures/Architekturmodell.png}
    \caption{Architekturmodell der Konsolenapplikation}
    \label{fig:architekturmodell}
\end{figure}

Abbildung \ref{fig:komponentendiagramm} stellt das Komponentenmodell der Applikation dar.

\begin{figure}[h]
    \centering
    \includegraphics[width=\textwidth]{pictures/ComponentDiagram.png}
    \caption{Komponentenmodell der Konsolenapplikation}
    \label{fig:komponentendiagramm}
\end{figure}


\section{FZ 1.2 Theoriebildung zur Vorverarbeitung medizinischer Texte}
\label{sec:FZ1.2} 

Um die Vorverarbeitung medizinischer Texte durchführen zu können, wird dem Präprozessor (Abbildung \ref{fig:preprocessor}) der Originaltext übergeben. Die Umwandlung und Ausgabe des Textes erfolgt über die Methode get\_preprocessed\_text(). Der Text wird in vollständige Sätze umgewandelt, die für sich alleine stehen können. D.h. Aufzählungen werden in Sätze umgewandelt und Pronomen durch konkrete Substantive ersetzt. 

Für die Erkennung der Satzeinheiten kann die Bibliothek \emph{pySBD - python Sentence Boundary Disambiguation} verwendet werden, die regelbasiert auf Grundlage von 100 \glqq Goldenen Regeln\grqq{} arbeitet.

Falls möglich, soll auch eine Indexstruktur (Kapitel-/Abschnittsüberschrif\-ten) ermittelt werden.

\begin{figure}[h]
    \centering
    \includegraphics[width=\textwidth]{pictures/RuleBasedPreprocessor.png}
    \caption{Klassendiagramm des Präprozessors der Konsolenapplikation}
    \label{fig:preprocessor}
\end{figure}

\section[FZ 2.3 Theoriebildung zur Überf. med.Fachvokabulars]{FZ 2.3 Theoriebildung zur Überführung medizinischen Fachvokabulars in maschinenlesbare Form zur weiteren Verarbeitung durch NLP}
\label{sec:FZ2.3} 

Beim Starten des MedExtractors wird eine KnowledgeExtractor-Instanz erzeugt (Abbildung \ref{fig:KnowledgeExtractor}). Zur Instanziierung wird der Dateiname der persistent gespeicherten Wissensbasis übergeben. Das KnowledgeExtractor-Objekt verwaltet das KnowledgeBase-Objekt. Der EntityRuler wird der spaCy-Pipeline hinzugefügt.

Dieser Instanz kann durch den Aufruf KnowledgeExtractor(text:String) ein vom Preprocessor vorbereiteter Abschnitt oder Satz zur Analyse übergeben werden. Daher muss KnowledgeExtractor die Methode \_\_call\_\_(text:String) implementieren. Zusätzlich kann über die Methode set\_context(Entity[]) dem KnowledgeExtractor mitgeteilt werden, um welches Thema es in dem Abschnitt oder Satz geht. Das Thema ergibt sich aus Entitäten, die in Kapitel- oder Abschnittsüberschriften gefunden werden.

is\_related() ist eine private Methode, die vom KnowledgeExtractor-Objekt verwendet wird, um zu prüfen, ob zwei beliebige vom EntityRuler gefundene Entitäten in Beziehung zueinander stehen. Die SemanticRelation-Objekte werden auf Basis eines wiederholten Aufrufs von is\_related() mit unterschiedlichen Kombinationen gefundener Entitäten erzeugt. 

\begin{figure}[h]
    \centering
    \includegraphics[width=\textwidth]{pictures/KnowledgeExtractor.png}
    \caption{Klassendiagramm des KnowledgeExtractors der Konsolenapplikation}
    \label{fig:KnowledgeExtractor}
\end{figure}

\section{FZ 3.2 Theoriebildung zur Wissensrepräsentation}
\label{sec:FZ3.2} 

Der RDFSerialiser (Abbildung \ref{fig:RDFSerialiser}) serialisiert die Wissensrepräsentation in eine RDF/XML-Datei.
Dabei arbeitet der RDFSerialiser mit der Python-Biblio\-thek rdflib. Zunächst soll ein Graph ohne Inhalt erstellt werden mit der Funktion create\_graph, dies geschieht mit der Klasse GraphManager. Anschließend sollen die Inhalte der übergebenen KnowledgeBase in den Graphen übertragen werden (knowledgebase\_to\_graph), dies geschieht mit den für GraphManager implementierten Methoden. Der so entstandene Graph wird dann mit der Methode serialise\_graph in das in serialisation\_format spezifizierte Format übertragen und anschließend die so entstandene Wissensrepräsentation gespeichert.

\begin{figure}[h]
    \centering
    \includegraphics[width=\textwidth]{pictures/RDFSerialiser.png}
    \caption{RDFSerialiser der Konsolenapplikation}
    \label{fig:RDFSerialiser}
\end{figure}


\section{Zusammenfassung}
\label{sec:zusammenfassung modellierung} 

Abbildung \ref{fig:mainClassDiagram} stellt das Hauptklassendiagramm der Applikation dar.

\begin{figure}[h]
    \centering
    \includegraphics[width=\textwidth]{pictures/Main.png}
    \caption{Hauptklassendiagramm der Konsolenapplikation}
    \label{fig:mainClassDiagram}
\end{figure}