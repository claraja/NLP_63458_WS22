% !TEX root =expose.tex
\chapter{Planung}

In diesem Kapitel wird die vorläufige Gliederung und Zeitplanung des Fachpraktikums vorgestellt.

\section{Gliederung}

\begin{enumerate}
\item Einleitung

\begin{enumerate}[label=\theenumi.\arabic*]
\item Motivation
\item Problembeschreibung
\item Forschungsfragen
\item Forschungsmethode
\item Forschungsziele
\end{enumerate}

\item Stand der Wissenschaft

\begin{enumerate}[label=\theenumi.\arabic*]
\item FZ 1.1 Aufbau und Struktur medizinischer Texte
\item FZ 2.1 Überblick über depressive Erkrankungen
\item FZ 2.2 Automatisierung durch NLP
\item FZ 3.1 Wissensrepräsentation mittels RDF
\end{enumerate}

\item Theoriebildung

\begin{enumerate}[label=\theenumi.\arabic*]
\item FZ 1.2 Theoriebildung zur Vorverarbeitung medizinischer Texte
\item FZ 2.3 Theoriebildung zur Überführung medizinischen Fachvokabulars
in maschinenlesbare Form zur weiteren Verarbeitung durch NLP
\item FZ 3.2 Theoriebildung zur Wissensrepräsentation
\end{enumerate}

\item Implementierung

\begin{enumerate}[label=\theenumi.\arabic*]
\item FZ 1.3 Implementierung zur Vorverarbeitung medizinischer Texte
\item FZ 2.4 Implementierung zur Überführung medizinischen Fachvokabulars
in maschinenlesbare Form zur weiteren Verarbeitung durch NLP
\item FZ 3.3 Implementierung zur Wissensrepräsentation
\end{enumerate}

\item Evaluation

\begin{enumerate}[label=\theenumi.\arabic*]
\item FZ 1.4 Evaluation zur Vorverarbeitung medizinischer Texte
\item FZ 2.5 Evaluation zur Überführung medizinischen Fachvokabulars
in maschinenlesbare Form zur weiteren Verarbeitung durch NLP
\item FZ 3.4 Evaluation zur Wissensrepräsentation
\item Zusammenfassung
\end{enumerate}

\item Zusammenfassung und Diskussion

\begin{enumerate}[label=\theenumi.\arabic*]
\item Ergebnisse
\item Offene Fragen

\end{enumerate}
\end{enumerate}

\begin{itemize}
\item Anhang
\item Literatur
\end{itemize}
