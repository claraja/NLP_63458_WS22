Im Folgenden sollen die Fälle untersucht werden, in denen der Medextractor falsche Symptom-Krankheit-Zuordnungen findet. Die vom Medextractor analysierten Texte stammen dabei aus drei Quellen:

\begin{enumerate}
	\item Webseite des National Health Service UK (NHS) - 49 Texte zu mehreren mentalen Erkrankungen oder Störungen (Agoraphobie, Alkoholmissbrauch, Angststörungen, Essstörungen, Demenz, Depression, Panikstörungen, u.s.w.) \cite{nhs_webpage}
	\item Der Artikel zum Thema Mental Disorder auf Wikipedia \cite{wikimentaldisorder}
	\item Clinical Handbook of Psychological Disorders \cite{clinicalhandbook}
\end{enumerate}

Zur Auswertung dienen die xml-Dateien für den Entity-Linker, da diese auch die Sätze enthalten, in denen Krankheiten bzw. Symptome gefunden wurden. Die Anzahl gefundener Krankheiten, Symptome und Trainingssätze lässt sich einfach durch Zählen der Tags <\textbackslash entity>, <\textbackslash alias> und <\textbackslash sample> ermitteln. Das Ergebnis zeigt Tabelle \ref{tab:zaehlung}. Entitäten bedeuten hierbei Krankheiten und Aliase bedeuten Symptome.

\begin{table}
\begin{center}
\begin{tabular}{lrrrr}
\hline
\textbf{Quelle}	& \textbf{Dateigröße}	& \textbf{Entitäten} & \textbf{Aliase} & \textbf{Sätze} \\
\hline
NHS &	272 KB & 93 & 221 & 262 \\
Wikipedia & 78 KB & 55 & 67  & 62 \\
Handbook & 3,5 MB & 204 & 485  & 917 \\
\hline
\end{tabular}
\caption{Von Medextractor gefundene Entitäten, Aliase und Beispielsätze}
\label{tab:zaehlung}
\end{center}
\end{table}

