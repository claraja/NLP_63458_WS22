\subsection{Stärken und Schwächen des Preprocessors}
\label{sec:evaluation_preprocessor}
\subsubsection{Funktionen}
[TODO]

\subsubsection{Einschränkungen} 
Bei einigen Formatierungen von Sätzen / Informationsauflistungen hat die aktuelle Implementierung des Preprocessors noch Probleme.
So kann zum Beispiel der folgende (verkürzte) Abschnitt einer Informationsseite zum Thema \emph{Drug-Dependent People} (\cite{drug_dependent_people}) noch nicht in sinnvolle Sätze transformiert werden:\\
\begin{addmargin}{10pt}
	\emph{
		All drug dependency exhibits similar characteristics— • Chronic, progressive, and relapsing disease • Denial • Lying and deceit • Changes in normal behavior
	}
\end{addmargin}
\vspace*{5mm}

\subsection{PySBD}
\label{evaluation_pysbd}
PySBD (\cite{sadvilkar_pysbd_2020}) ist ein regelbasierter \emph{Sentence Boundary Detector/Disambiguater} (Satzgrenzen-Finder). Golden Rules Set (\cite{golden_rules}) ...[TODO]

\subsubsection{Vergleich der Einteilung von Texten in Sätze mit und ohne PySBD}
Es soll untersucht werden worin die Vor- und Nachteile beim Einteilen von Texten in Sätze mit der Bibliothek \emph{PySBD} liegen. 
Alternativ gibt es die Möglichkeit den Sentencizer von SpaCy zu nutzen. Für den Vergleich wird die Verarbeitung der 51 Texte in 
\emph{medextractor/resources/to\_analyze} verglichen. Dies geschieht in der python-Datei \emph{medextractor/preprocessor/pysbd\_evaluation\_sentences.py}.\\

Es fällt schnell auf, dass der Hauptunterschied in der Einteilung von Texten in einzelne Sätze mit und ohne PySBD darin besteht, 
dass ein Zeilenumbruch unterschiedlich interpretiert wird.
Aus dem folgenden Textabschnitt in der Datei \emph{agorophobia.txt}\\

\begin{quotation}
	\glqq \verb!\n!Symptoms - Agoraphobia\verb!\n!\verb!\n!The severity of agoraphobia can vary significantly between individuals.\grqq
\end{quotation}

erhält man mit dem SpaCy Sentencizer zum Beispiel den Satz

\begin{quotation}
	\glqq \verb!\n!Symptoms - Agoraphobia\verb!\n!\verb!\n!The severity of agoraphobia can vary significantly between individuals.\grqq,
\end{quotation}

während die Aufteilung in Sätze mit PySBD folgendermaßen aussieht:

\begin{enumerate}
	\item \glqq Symptoms - Agoraphobia\verb!\n!\verb!\n!\grqq
	\item \glqq The severity of agoraphobia can vary significantly between individuals.\verb!\n!\verb!\n!\grqq
\end{enumerate}

Insgesamt wird von den 51 Texten nur einer vom PySBD- und SpaCy-Sentencizer auf die gleiche Art in Sätze eingeteilt.\\
Wendet man nun die folgende Art der Nachverarbeitung der Texte an:
\begin{enumerate}
	\item Sätze, die ein \glqq \verb!\n!\grqq enthalten, werden an dieser Stelle in zwei Sätze aufgesplittet,
	\item Leerzeichen am Anfang und am Ende von Sätzen werden entfernt,
	\item leere Sätze werden entfernt,
\end{enumerate}
 werden schon 41 der 51 Texte komplett gleich verarbeitet. Ausnahmen bilden dann (fast) nur noch Sätze, die Zeichen wie z.B. \glqq ]\grqq, \glqq :\grqq und \glqq -\grqq enthalten (diese speziellen Zeichen werden vom PySBD- und SpaCy-Sentencizer unterschiedlich behandelt), oder Satzanfänge ohne vorhergehendes Leerzeichen.\\

 
\subsubsection{Fazit zu PySBD}
Der Preprocessing-Schritt des MedExtractors enthält nicht nur das Erkennen von Satzgrenzen, sondern noch weitere Schritte um die Sätze so umzuformen, dass sie in der weiteren Verarbeitung mit dem MedExtractor gut genutzt werden können (siehe Kapitel [TODO]). Für die Zwecke des MedExtractors und für die zur Evaluierung genutzten Texte war PySBD sehr hilfreich. Während der Evaluierung ist aber auch aufgefallen, dass mit Berücksichtigung der Eigenheiten des SpaCy Sentencizers wahrscheinlich ähnlich gute Ergebnisse im Preprocessing-Schritt hätten erzielt werden können. Somit kann PySBD für die Zwecke des MedExtractors zwar empfohlen werden, mit anderen Sentence Boundary Detectorn könnten aber voraussichtlich ähnlich gute Ergebnisse erzielt werden. \\
In der weiteren Entwicklung des Preprocessors, bzw. des MedExtractors, könnte man versuchen mit einer besseren Einstellung der Parameter von PySBD bessere Ergebnisse zu erzielen, oder mit einem anderen \emph{Sentence Boundary Detector} (mehrere Alternativen werden in \cite{sadvilkar_pysbd_2020} genannt) zu arbeiten.

